\section{Prototyp}
\label{sec:prototype}

\subsection{Auswahl des Frameworks}
    Ein (kurzer) Abschnitt der Arbeit soll die Auswahl des genutzen Frameworks für den Prototypen darstellen, da dies eine zentrale Entscheidung für die Arbeit und somit auch essentiell für die Bearbeitung der Problemstellung ist.
    Es existieren mittlerweile diverse Frameworks mit verschiedenen Anwendungsgebieten, daher sollte das letztendlich verwendete Framework folgenden Kriterien entsprechen:
    \begin{itemize}
        \item Open Source \textrightarrow\ eigene Einflussmöglichkeit
        \item aktive Community \textrightarrow\ Unterstützung bei Schwierigkeiten
        \item aktive Weiterentwicklung \textrightarrow\ zumindest in naher Zukunft besteht die Möglichkeit, dass das Framework auch in Produkten auf dem Markt genutzt werden könnte
        \item Auswahl an Sicherheitsmechanismen \textrightarrow\ Evaluierung von verschiedenen Konfigurationen möglich
        \item Theoretische Möglichkeit des Schließens von den in der Analyse erarbeiteten Sicherheitslücken
    \end{itemize}
    \medskip
    \textbf{Auswahl verfügbarer Frameworks nach erster Recherche}
    \begin{itemize}
        \item Hyperledger (Auswahl an verschiedenen Projekten mit unteschiedlichen Einsatzszenarien):\\
            Favorit: Iroha - mit rollenbasierter Rechtevergabe und Unterstützung für mobile Plattformen, sowie Bibliotheken für mehrere Sprachen und guter Dokumentation; Composer, viele Beispiele, aktive Community
        \item Exonum
        \item Ethereum
        \item Multichain
    \end{itemize}



\subsubsection{Architektur und Funktionsweise}
    Enstprechend dem Kapitel mit dem Stand der Wissenschaft wurden die Konzepte im Framework umgesetzt.
    Ausgehend von der Dokumentation des Frameworks \cite{SoramitsuCo.} wurde die Architektur ausgearbeitet. 
    
    
    Keine Standardisierung von 