\section{Einleitung}\todo[color=cyan]{Vorwort und Abstract schreiben}
    Mit stetig zunehmender Vernetzung des Lebens ist davon auch häufig der eigene Wohnraum betroffen. 
    Der Trend zu sogenannten Smart Homes ist klar erkennbar. 
    Von Küchengeräten über Beleuchtung, Sprinkleranlagen im Garten und Ga\-ra\-gen\-tü\-ren - immer mehr Geräte werden mit einem Netzwerk und gar mit dem Internet zu einem sogenannten \gls{iot} verbunden. 
    Gesteurert wird dies meist mit dem Smartphone entweder direkt oder über ein spezielles Hub, über das alle Informationen zentral fließen.
    So auch Smart Locks (wörtl. ,,intelligente Schlösser``).

    Diese werden häufig auch bei Buchung und Vermietung von privaten Unterkünften oder im eigenen Heim an der als Türschlosser oder auch in Form von Vorhängeschlössern eingesetzt und sollen den Besitzern die Möglichkeit bieten schlüssellos und bequem mittels Smartphone das Schloss zu öffnen und zu schließen.
    Häufig bieten Smart Locks auch Funktionen zur Administration von Berechtigungen, wie beispielsweise bestimmte Nutzer zeitweise dazu zu berechtigen das Türschloss zu öffnen und zu schließen.
    Oft wird zur Übertragung der Signale Bluetooth Low Energy verwendet.
    
    Ebenfalls im Trend liegt die Technologie der Blockchain, welche mit dem Erfolg der Kryptowährung Bitcoin nun auch in anderen Gebieten wie im Internet of Things und im Smart Home Anwendung findet.
    Da im Smart Home häufig auch kritische Daten, wie beispielsweise personenbezogene Daten ausgetauscht werden, ist deren Sicherheit zu garantieren wichtig.
    Ein zentrales Merkmal der Blockchain ist die Dezentralisierung der ,,Buchführung`` von Transaktionen.
    \newline
    
    \noindent Aufgrund vermehrter Berichte über Sicherheitsvorfälle bei \gls{iot}-Geräten ist es umso nötiger die Sicherheit der im \gls{iot} verarbeiteten Daten und die Funktion der vernetzten Geräte zu gewährleisten.
    Diese Berichte umfassen Schwachstellen wie hardcoded Schlüssel, im Klartext gespeicherte Passwörter, Möglichkeit für Replay-Angriffe, Device-Spoofing\cite{Rose2016} und ungesicherte APIs beim Kommunikation mit der Cloud\cite{Stykas2018}. 
    Als eine der schwerwiegensten Schwachstellen wird außerdem die Zentralisierung von \gls{iot}-Geräten vor allem in der Cloud beschrieben\cite{Kshetri2017}.
    \newline\smallskip
    
    \noindent\textbf{Motivation}\todo[color=cyan]{motivation schreiben}\newline
    The key thing to keep in mind is this: if you have a set of users (a) who want to trade digital tokens, and (b) have agreed on how these tokens are generated, then a blockchain network is an ideal tool to use both for exchanging these tokens, and tracking who has what. No middleman is needed to facilitate the exchanges cause every node on the network runs the the necessary checks and reaches consensus on the accepted result. Asset tracking comes out-of-the-box since every node has access to the agreed set of cryptographically verifiable transactions on the blockchain.\cite{Christidis2016}
    \newline\smallskip
    
\section{Problemstellung}
    %Anhand welcher Theorien, Methoden und Vorgehensweisen erledigt die Bachelorarbeit diese Aufgabe?
    Gerade bei Smart Locks ist es unbedingt nötig diese Schwachstellen zu unterbinden.
    Durch das oben erwähnte dezentrale Konzept der Blockchain\cite{Nakamoto2008} lohnt es sich diese Technologie im Kontext des \gls{iot}, am Beispiel von Smart Locks zu untersuchen.
    Zudem verspricht man sich auch im Bereich des Identitäts- und Zugriffsmanagements von dem Konzept Blockchain Angriffe wie Device-Spoofing durch das Speichern von Gerätesignaturen zu unterbinden\cite{Kshetri2017}.
    \newline
    \noindent Als Ziel der Arbeit soll die Frage erörtern ob die Block\-chain\--Tech\-no\-lo\-gie aus dem Aspekt der Sicherheit dafür geeignet ist, im Bereich der Smart Locks (und erweitert im Bereich Smart Home) eingesetzt zu werden.
    Dies soll mit Hilfe eines Prototypen eines Smart Locks untersucht werden.
    
    Der Fokus des Prototypen soll dabei aber nicht auf der Umsetzung der Hardware liegen, sondern auf der Nutzung eines aktuell vorhandenen Frameworks, also auf aktuell plausiblen Implementierungen. 
    Primär sollen bereits publizierte Schwachstellen bei Smart Locks analysiert werden und bei der Umsetzung des Prototypen vermieden werden. 
    Dies wird nach fertigstellung des Prototypen untersucht.
    Je nach Ergebnis lässt sich dann auf die Kernfrage schließen.

    \subsection{Methodik}
        Zunächst sollen bekannte Schwachstellen aktueller Produkte analysiert werden.
        Als roter Faden der Analyse werden die \gls{owasp}-Top10 für das \gls{iot}\cite{Miessler2015a} verwendet.
        Im Anschluss wird als erstes ein passendes Framework ausgewählt, welches theoretisch die in der Analyse gefundenen Lücken schließen könnte.
        Auf Basis dieses Frameworks wird dann der Prototyp entworfen und umgesetzt.
        Der Prototyp soll ebenfalls anhand der \gls{owasp}-Top10 evaluiert werden.
        Danach wird zwsichen den beim Prototyp gefundenen und den zuvor bei aktuellen Produkten analysierten Schwachstellen verglichen.
        Dies geschieht mit Hilfe des \gls{cvss}-Bewertungsschemas, welches eine Vergleichbarkeit zwischen den gefundenen Lücken schaffen soll.
        Abschließend wird die Problemstellung mittels des Vergleichs erörtert.

\vspace{3em}    
\noindent In \fref{sec:sota} werden zunächst die Grundlagen für diese Arbeit vorgestellt, darunter das Konzept einer Blockchain in \fref{sec:blockchain_introduction}, das Double-Spending Problem in \fref{sec:blockchain_doublespend} und einige Sicherheitsaspekte in \fref{sec:blockchain_security}.
Weiterhin werden \gls{iot}(\fref{sec:iot}), Smart Home(\fref{sec:smart_home}) und Smart Locks(\fref{sec:smart_locks}) erklärt\todo[color=orange]{wdh}.
Auf Sichereitsanalysen im \gls{iot} wird ausführlicher eingegangen.
\todo[color=yellow]{kurzen Überblick über die Arbeit schreiben}
    