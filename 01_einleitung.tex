\section{Einleitung}
    Mit stetig zunehmender Vernetzung des Lebens ist davon auch häufig der eigene Wohnraum betroffen. 
    Der Trend zu sogenannten Smart Homes ist klar erkennbar. 
    Von Küchengeräten über Beleuchtung, Sprinkleranlagen im Garten und Ga\-ra\-gen\-tü\-ren - immer mehr Geräte werden mit einem Netzwerk und gar mit dem Internet zu einem sogenannten \gls{iot} verbunden. 
    Gesteurert wird dies meist mit dem Smartphone entweder direkt oder über ein spezielles Hub, über das alle Informationen zentral fließen.
    So auch Smart Locks (wörtl. ,,intelligente Schlösser``).

    Diese werden häufig auch bei Buchung und Vermietung von privaten Unterkünften oder im eigenen Heim an der als Türschlosser oder auch in Form von Vorhängeschlössern eingesetzt und sollen den Besitzern die Möglichkeit bieten schlüssellos und bequem mittels Smartphone das Schloss zu öffnen und zu schließen.
    Häufig bieten Smart Locks auch Funktionen zur Administration von Berechtigungen, wie beispielsweise bestimmte Nutzer zeitweise dazu zu berechtigen das Türschloss zu öffnen und zu schließen.
    Oft wird zur Übertragung der Signale Bluetooth Low Energy verwendet.
    
    Ebenfalls im Trend liegt die Technologie der Blockchain, welche mit dem Erfolg der Kryptowährung Bitcoin nun auch in anderen Gebieten wie im Internet of Things und im Smart Home Anwendung findet.
    Da im Smart Home häufig auch kritische Daten, wie beispielsweise personenbezogene Daten ausgetauscht werden, ist deren Sicherheit zu garantieren wichtig.
    Ein zentrales Merkmal der Blockchain ist die Dezentralisierung der ,,Buchführung`` von Transaktionen.\newline
    
    Aufgrund vermehrter Berichte über Sicherheitsvorfälle bei \gls{iot}-Geräten ist es umso nötiger die Sicherheit der im \gls{iot} verarbeiteten Daten und die Funktion der vernetzten Geräte zu gewährleisten.
    Diese Berichte umfassen Schwachstellen wie hardcoded Schlüssel, im Klartext gespeicherte Passwörter, Möglichkeit für Replay-Angriffe, Device-Spoofing\cite{Rose2016} und ungesicherte APIs beim Kommunikation mit der Cloud\cite{Stykas2018}. 
    Als eine der schwerwiegensten Schwachstellen wird außerdem die Zentralisierung von \gls{iot}-Geräten vor allem in der Cloud beschrieben\cite{Kshetri2017}.
    
    
    
    
    
    \subsection*{\textsc{Draft}}
    \textbf{Motivation}\linebreak
    The key thing to keep in mind is this: if you have a set of users (a) who want to trade digital tokens, and (b) have agreed on how these tokens are generated, then a blockchain network is an ideal tool to use both for exchanging these tokens, and tracking who has what. No middleman is needed to facilitate the exchanges cause every node on the network runs the the necessary checks and reaches consensus on the accepted result. Asset tracking comes out-of-the-box since every node has access to the agreed set of cryptographically verifiable transactions on the blockchain.\cite{Christidis2016}