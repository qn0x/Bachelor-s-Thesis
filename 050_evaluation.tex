\section{Evaluation des Prototypen}
\label{sec:evaluation}
    Wie auch in \fref{sec:analysis} bei der Analyse der Schwachstellen von aktuellen Produkten, wird auch die Analyse des Prototypen nach \cite{Miessler} gemacht. 
    Aufgrund dessen, dass das evaluierte System noch im Status eines minimalen Prototypen ist, fallen die Möglichkeiten eine Web- oder Mobileoberfläche zu testen weg. 
    Ebenfalls ist es nicht möglich die physische Sicherheit zu testen.
    \medskip\\
    Mittels des Frameworks wird eine \gls{rest}-\gls{api} generiert, die für die Evauluation des Prototypen bei Bedarf verwendet wird. 
    Diese wird so konfiguiert, dass deren Verwendung eine Authentifizierung voraussetzt, sowie jeglicher Nachrichtenaustausch mittels \gls{tls} verschlüsselt wird.

    \subsection{Vulnerability Assessment}
        \begin{itemize}[noitemsep]
            \item Bei Initialisierung manipulieren der Daten, um die Besitzerrolle zu erlangen
            \item als vendor ausgeben und netzwerk zurücksetzen
        \end{itemize}
    
        \subsubsection*{I1: Insecure Web Interface}
            Da der Prototyp kein Webinterface hat, kann dieser Aspekt für die Evauluation nicht betrachtet werden.
           
        \subsubsection*{I2: Insecure Authentification/Authorization}
            \begin{itemize}[leftmargin=0cm,label={}]
                \item \emph{Privilege Escalation}\label{vuln:prototype_pe}\\
        	        Stuff.
    	        \item \emph{ID-Card theft}
    	        \item \emph{Single Point of Failure}\label{vuln:prototype_spf}\\
                    In Composer gibt es zwei unterschiedliche Arten von Adminsitratoren: den Business Network Administrator (je einen pro Business Network) und den Peer Administrator (je einen pro Hyperledger Fabric Instanz)
            \end{itemize}
            
        \subsubsection*{I3: Insecure Network Services}
            \begin{itemize}[leftmargin=0cm,label={}]
                \item \emph{Denial of Service via \gls{rest}-Server}\label{vuln:prototype_dos}\\
        	        Jeder Knoten kann einen \gls{rest}-Server für seine Nutzer bereitstellen. 
        	        Jedoch wird bei der Authentifizierung kein Rate-Limiting angewendet. 
        	        Bei Knoten, die entsprechend wenig Rechenleistung bieten kann kann dies unter Umständen zu einem \gls{dos} führen. 
        	        Zwar wird dadurch dem Nutzer der Zugriff zum Service verweigert, das Netzwerk an sich bleibt jedoch, wenn genug weitere Knoten nicht betroffen sind, funktional in Betrieb.
            \end{itemize}
            
        \subsubsection*{I4: Lack of Transport Encryption/Integrity Verification}
            Für diese Kategorie wurden keine Schwachstellen gefunden. 
            Die Kommunikation mit zwischen den Knoten, sowie dem Client und Knoten via \gls{rest}-Server kann mittels \gls{tls} gesichert werden. 
            
        \subsubsection*{I5: Privacy Concerns}
            \begin{itemize}[leftmargin=0cm,label={}]
                \item \emph{Unprotected User Information}\label{vuln:prototype_userdata}\\
                    Im Prototypen müssen für die validierung der Transaktionen alle Nutzer alle anderen Nutzer sehen. 
                    Dazu muss in den \gls{acl}-Regeln die Operation \colorbox{light-gray}{\lstinline{READ}} auf alle Teilnehmertypen für alle Teilnehmer erlaubt sein. 
                    Der Mechanismus der Regeln im Hyperledger Composer Framework erlaubt es jedoch nicht Regeln für die einzelnen Attribute jeder Ressource festzulegen. 
                    Dazu gehören im Fall des Typs \colorbox{light-gray}{\lstinline{User}} auch personenbezogene Daten wie Vor- und Nachname und E-Mailadresse.\\
                    Betrachtet man das Beispielnetzwerk in \fref{fig:pt_network} aus \fref{sec:prototype_arch_concept}, so könnte der Hersteller \colorbox{light-gray}{\lstinline{Vendor 1}} auch die Daten der Kunden von \colorbox{light-gray}{\lstinline{Vendor 2}}, in diesem Fall \colorbox{light-gray}{\lstinline{User 2}} einsehen und speichern.
            \end{itemize}
            
        \subsubsection*{I6: Insecure Cloud Interface}
            Als Cloud Interface wird hier die Verbindung von einem Peer zum anderen betrachtet.
            Die Daten aus den Transaktionen werden werden sozusagen ,,hochgeladen`` und der Nutzer kann nicht genau sagen auf welchem Server bzw. Knoten diese letztendlich auch gespeichert werden-
            \begin{itemize}[leftmargin=0cm,label={}]
                \item \emph{Participant Enumeration}\label{vuln:prototype_enum_part}
                    Erstellt ein autorisierter Nutzer einen neuen Teilnehmer (der Typ des Nutzers ist irrelevant), so wird diesem automatisch eine ID zugewiesen. 
                    Diese ID besteht standardmäßig aus lediglich vier Ziffern und ist somit für einen Angreifer einfach zu erraten.
                \item \emph{Asset Enumeration}\label{vuln:prototype_enum_asset}
                    Wie auch bei der Participant Enumeration werrd bei der Erstellung von Assets die IDs automatisch vergeben. 
                    Auch hier besteht die ID lediglich aus viel Ziffern.
            \end{itemize}
            
        \subsubsection*{I7: Insecure Mobile Interface}
            Da der Prototyp kein mobiles Interface hat, kann dieser Aspekt nicht betrachtet werden.
            
        \subsubsection*{I8: Insufficient Security Configurability}
            \begin{itemize}[leftmargin=0cm,label={}]
    	        \item \emph{Unencrypted Data Storage}\label{vuln:prototype_db}\\
                    Die Speicherung der jeweiligen Kopie der Blockchain auf den Knoten erfolgt über eine unverschlüsselte Datenbank. 
                    Hat ein Angreifer Zugriff auf den Knoten und die Datenbank (der Angreifer muss dafür nicht zwingend Teilnehmer im Netzwerk sein), so ist es für ihn möglich an alle Daten, die auf der Blockchain gepeichert sind, zu kommen. 
                    Da der Prototyp mit dem Konzept einer privaten Blockchain funktioniert, sollten die dort gespeicherten Daten auch nicht an Außenstehende gelangen.
            \end{itemize}
            
        \subsubsection*{I9:Insecure Software/Firmware}
            Als Firmware wird das unter Hyperledger Composer liegende Fabric-Framework aufgefasst. 
            
            \begin{itemize}[leftmargin=0cm,label={}]
    	        \item die Fabric
            \end{itemize}
            
        \subsubsection*{I10: Poor Physical Security}
            Wird auch hier wie in \fref{sec:problem_limit} erwähnt, nicht betrachtet.
    
    \sloppy\url{https://wiki.hyperledger.org/_media/security/technical_report_linux_foundation_fabric_august_2017_v1.1.pdf}\\
    \sloppy\url{https://wiki.hyperledger.org/_media/security/management_report_linux_foundation_fabric_august_2017_v1.1.pdf}
    
    \subsection{Bewertung nach CVSS}
    	\begin{table}[H]
            \centering
            \begin{tabular}{|m{0.2\textwidth}|m{0.2\textwidth}|}
            \hline
            \textbf{Rating}   & \textbf{\gls{cvss} Score}   \\ \hline
            \rowcolor{light-gray}
            None              & 0.0                         \\ \hline
            Low               & 0.1 - 3.9                   \\ \hline
            \rowcolor{light-gray}
            Medium            & 4.0 - 6.9                   \\ \hline
            High              & 7.0 - 8.9                   \\ \hline
            \rowcolor{light-gray}
            Critical          & 9.0 - 10.0                  \\ \hline
            \end{tabular}
            \caption[CVSS Scores des Prototypen]{CVSS Scores des Prototypen}
            \label{tab:eval_cvss}
        \end{table}
