\section{Evaluation des Prototypen}
\label{sec:evaluation}

    \subsection{Vulnerability Assessment des Prototypen}

        \begin{itemize}[noitemsep]
            \item Evaluierung nach \cite{Miessler}, aber Fokus auf die in \fref{sec:prototype_requirements} ausgewählten Schwachstellen. 
            \item Vorgehen für jeden Punkt bei der Evaluierung beschreiben und Ergebnisse festhalten. 
            \item Falls zusätzliche Schwachstellen auffallen, diese auch mit einbeziehen.
        \end{itemize}
        
        \begin{itemize}[noitemsep]
            \item Speicher vollaufen lassen, indem viele Transaktionen getätigt werden \textrightarrow\ Fehlerzustand?
            \item Bei Initialisierung manipulieren der Daten, um die Besitzerrolle zu erlangen
            \item Enumeration, da im aktuellen Zustand alle IDs von Hand erstellt werden können \textrightarrow\ kann über API ausgenutzt werden oder über Webseite: wenn Erstellen nicht möglich, dann hat man als Angreifer eine ID, die auch genutzt wird und somit Informationen gewonnen
            \item als vendor ausgeben und netzwerk zurücksetzen
        \end{itemize}
    
        \subsubsection*{I1: Insecure Web Interface}
           \begin{itemize}[leftmargin=0cm,label={}]
                \item \emph{Insecure API: User Enumeration}\label{vuln:prototype_userenum}\\
                    Stuff.
           \end{itemize} 
           
        \subsubsection*{I2: Insecure Authentification/Authorization}
            \begin{itemize}[leftmargin=0cm,label={}]
                \item \emph{Insufficient Password Protection}\label{vuln:prototype_pwdprot}\\
        	        Stuff.
            \end{itemize}
            
        \subsubsection*{I3: Insecure Network Services}
            \begin{itemize}[leftmargin=0cm,label={}]
                \item \emph{Denial of Service via Commands}\label{vuln:prototype_doscmd}\\
        	        Stuff.
            \end{itemize}
            
        \subsubsection*{I4: Lack of Transport Encryption/Integrity Verification}
            \begin{itemize}[leftmargin=0cm,label={}]
                \item \emph{Insecure Password Transmission}\label{vuln:prototype_pwdtrns}\\
                    sfaf
            \end{itemize}
            
        \subsubsection*{I5: Privacy Concerns}
            \begin{itemize}[leftmargin=0cm,label={}]
                \item \emph{Unprotected User Information}\label{vuln:prototype_userdata}\\
                    Im Prototypen müssen für die validierung der Transaktionen alle Nutzer alle anderen Nutzer sehen. 
                    Dazu muss in den \gls{acl}-Regeln die Operatrion \colorbox{light-gray}{\lstinline{READ}} auf alle Teilnehmertypen für alle Teilnehmer erlaubt sein. 
                    Der Mechanismus der Regeln im Hyperledger Composer Framework erlaubt es jedoch nicht Regeln für die einzelnen Attribute jeder Ressource festzulegen. 
                    Dazu gehören im Fall des Typs \colorbox{light-gray}{\lstinline{User}} auch personenbezogene Daten wie Vor- und Nachname und E-Mailadresse. 
                    Betrachtet man das Beispielnetzwerk in \fref{fig:pt_network} aus \fref{sec:prototype_arch_concept}, so könnte der Hersteller \colorbox{light-gray}{\lstinline{Vendor 1}} auch die Daten der Kunden von \colorbox{light-gray}{\lstinline{Vendor 2}} einsehen und speichern.
            \end{itemize}
            
        \subsubsection*{I6: Insecure Cloud Interface}
            \begin{itemize}[leftmargin=0cm,label={}]
                \item 
            \end{itemize}
            
        \subsubsection*{I7: Insecure Mobile Interface}
            \begin{itemize}[leftmargin=0cm,label={}]
                \item \emph{Stolen User Settings}\label{vuln:prototype_usersettings}\\
                    sadfasdf
            \end{itemize}
            
        \subsubsection*{I8: Insufficient Security Configurability}
            Nicht mittels \gls{cvss} bewertbar
            
        \subsubsection*{I9:Insecure Software/Firmware}
            \begin{itemize}[leftmargin=0cm,label={}]
    	        \item \emph{Hardcoded Firmware Key}\label{vuln:prototype_firmwarekey}\\
                    asdfasdf
            \end{itemize}
            
        \subsubsection*{I10: Poor Physical Security}
            Wird auch hier wie in \fref{sec:problem_limit} erwähnt, nicht betrachtet.
    
    
    \subsection{Bewertung nach CVSS}
    	\todo[color=yellow]{not sure...}\noindent In die Tabelle die Strings oder einzelne Spalten mit den Faktoren etc.?\\
    	{\large\textsc{Tabelle unfertig, nur provisorisch}}
        %\begin{sidewaystable}[H]
          {\centering
            \begin{table}[H]
            {\small
            \begin{tabular}{|>{\raggedright}p{0.28\textwidth}|m{0.56\textwidth}|m{0.05\textwidth}|}
                \hline
                \textbf{}                                      & \textbf{Vector String}                        & \textbf{Score}   \\ \hline
                \rowcolor{light-gray}
                \textbf{API: User Enumeration}                 & CVSS:3.0/AV:P/AC:L/PR:N/UI:N/S:C/C:N/I:N/A:H  & 5.3              \\ \hline
                \textbf{Insufficient Password Protection}      & CVSS:3.0/AV:N/AC:H/PR:N/UI:N/S:U/C:L/I:H/A:N  & 6.5              \\ \hline
                \rowcolor{light-gray}
                \textbf{API: Privilege Escalation}             & CVSS:3.0/AV:N/AC:L/PR:L/UI:N/S:C/C:H/I:N/A:L  & 8.5              \\ \hline
                \textbf{Insecure Password Policy}              & CVSS:3.0/AV:N/AC:H/PR:N/UI:N/S:C/C:L/I:N/A:H  & 7.5              \\ \hline
                \rowcolor{light-gray}
                \textbf{Phishing}                              & CVSS:3.0/AV:N/AC:H/PR:N/UI:R/S:C/C:L/I:N/A:H  & 6.9              \\ \hline
                \textbf{Replay}                                & CVSS:3.0/AV:P/AC:L/PR:N/UI:R/S:U/C:N/I:H/A:L  & 4.9              \\ \hline
                \rowcolor{light-gray}
                \textbf{Device Spoofing}                       & CVSS:3.0/AV:P/AC:L/PR:N/UI:R/S:U/C:N/I:H/A:L  & 4.9              \\ \hline
                \textbf{Missing 2-Factor Authentication}       & CVSS:3.0/AV:N/AC:H/PR:N/UI:N/S:C/C:N/I:L/A:L  & 5.4
                \label{tab:evaluation_bewertung}\\ \hline
            \end{tabular}
            }
            \end{table}}
          %\caption{Nach \gls{cvss} bewertete Schwachstellen aus \fref{sec:analysis_vulns}}
        %\end{sidewaystable}
