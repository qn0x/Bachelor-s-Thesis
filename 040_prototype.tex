\section{Prototyp}
\label{sec:prototype}
    Der für diese erstellte Prototyp soll dem mininmalen Funktionsumfang als Beispiel für eine Implementierung des Anwendungsfalls einer Blockchain bei Smart Locks im \gls{iot} entsprechen. 
    Die demnach funktionalen Anforderungen werden in \fref{sec:prototype_func_req} festgelegt. 
    Es wird dennoch besonderer Wert auf die Sicherheit gelegt, indem die die Schwachstellen, die in \fref{sec:analysis} erläutert wurden, falls möglich, im Prototypen adressiert und mitigiert werden. 
    Dafür werden besondere Anforderungen, in \fref{sec:prototype_sec} gesondert erhoben. 
    Der Auswahlprozess des verwendeten Frameworks wird in \fref{sec:prototype_framework} knapp dargestellt. 
    Nach einer kurzen Einführung in das Framework, in der für den Prototypen relevante Eigenschaften, Funktionen und Komponenten erläutert werden, wird das Konzept des Prototypen und einige Implementierungsdetails vorgestellt (\fref{sec:prototype_arch}). 
