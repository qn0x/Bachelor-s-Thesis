\section{Prototyp}
\label{sec:prototype}

    Prototyp wurde schonmal von \cite{Han2017} entworfen, aber nur sehr oberflächlich.
    Basiert nicht darauf.

\subsection{Auswahl des Frameworks}
\label{sec:prototype_framework}
    Ein (kurzer) Abschnitt der Arbeit soll die Auswahl des genutzen Frameworks für den Prototypen darstellen, da dies eine zentrale Entscheidung für die Arbeit und somit auch essentiell für die Bearbeitung der Problemstellung ist.
    Es existieren mittlerweile diverse Frameworks mit verschiedenen Anwendungsgebieten, daher sollte das letztendlich verwendete Framework folgenden Kriterien entsprechen:
    \begin{itemize}
        \item Open Source \textrightarrow\ eigene Einflussmöglichkeit
        \item aktive Community \textrightarrow\ Unterstützung bei Schwierigkeiten
        \item aktive Weiterentwicklung \textrightarrow\ zumindest in naher Zukunft besteht die Möglichkeit, dass das Framework auch in Produkten auf dem Markt genutzt werden könnte
        \item Auswahl an Sicherheitsmechanismen \textrightarrow\ Evaluierung von verschiedenen Konfigurationen möglich
        \item Theoretische Möglichkeit des Schließens von den in der Analyse erarbeiteten Sicherheitslücken
    \end{itemize}
    \medskip
    \textbf{Auswahl verfügbarer Frameworks nach erster Recherche}
    \begin{itemize}
        \item Hyperledger (Auswahl an verschiedenen Projekten mit unteschiedlichen Einsatzszenarien):\\
            Favorit: Iroha - mit rollenbasierter Rechtevergabe und Unterstützung für mobile Plattformen, sowie Bibliotheken für mehrere Sprachen und guter Dokumentation; Composer, viele Beispiele, aktive Community
        \item Exonum
        \item Ethereum
        \item Multichain
    \end{itemize}
    Auswahl fiel auf Hyperledger Composer, da dieses Projekt am aktivsten Weiterentwickelt wird und reifer ist als Ihroha.
    Da Composer auf Fabric basiert, stellt es zur Konfiguration verschiedene Konsensalgorithmen zur Verfügung.

\subsection{Architektur und Funktionsweise}
\label{sec:prototype_functions}
    Enstprechend dem Kapitel mit dem Stand der Wissenschaft wurden die Konzepte im Framework umgesetzt.
    Ausgehend von der Dokumentation des Frameworks \cite{SoramitsuCo.} wurde die Architektur ausgearbeitet. 
    
    Keine Standardisierung und Best Practices von Architekturen, wie in fast allen Bereichen des \gls{iot}, daher eigentlich völlig frei.
    
    Notizen für den Prototypen:
    \begin{itemize}[noitemsep]
        \item Ein Asset als Token für (,,du darfst die Tür aufmachen``) je berechtigtem Nutzer \textrightarrow\ zurückverfolgbar (Accountability)
        \item Framework bietet in irgendeiner Form Access Control an?
        \item How to ensure a consistent state across all entities?...
        \item Vendor Server als Teilnehmer im Netzwerk oder nicht?
        \item Vertraulichkeit überhaupt irgendwie möglich oder durch das Konzept der Blockchain schon gar nicht?
        \item Pseudo-/Anonymisierung der Teilnehmer sinnvoll?
        \item Hyperledger biete CA für Identitäten, Authentifizierung
        \item 
    \end{itemize}
    
    Asset Types:
    \begin{itemize}[noitemsep]
        \item Door Key (kann zeitlich beschränkt sein), bei Öffnen an Schloss senden, bei Schließen wieder an Nutzer zurück \textrightarrow\ man kann sich sicher sein, dass die Tür bei 0 Token geschlossen ist.
        \item Token, der die Rolle repräsentiert?
        \item Um \gls{dos} zu vermeiden, etwas ähnliches wie Mutex-Token? Oder: wenn Tür
    \end{itemize}
    
    Participant Types:
    \begin{itemize}[noitemsep]
        \item Manufacturer
        \item Owner
        \item Guest
    \end{itemize}
    
    Transaction Types (entpricht den Funktionen, die man je nach Rolle ausführen darf):
    \begin{itemize}[noitemsep]
        \item OpenDoor
        \item User (AddUser, DeleteUser, ChangeUserRole)
        \item TimeSlot (AddTimeSlot, DeleteTimeSlot, ChangeGuestTimeSlot)
    \end{itemize}
    
    Rollenbasiertes Zugriffskonzept
    \begin{itemize}[noitemsep]
        \item 
    \end{itemize}
    