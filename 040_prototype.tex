\section{Prototyp}
\label{sec:prototype}
    Der für diese Untersuchung erstellte Prototyp soll als Beispiel für eine Implementierung des Anwendungsfalls einer Blockchain bei Smart Locks im \gls{iot} mit mininmalem Funktionsumfang entsprechen. 
    Die demnach erhobenen funktionalen Anforderungen werden in \fref{sec:prototype_func_req} festgelegt. 
    Es wird dennoch besonderer Wert auf die Sicherheit gelegt, indem die Schwachstellen, die in \fref{sec:analysis} erläutert wurden, falls möglich, im Prototypen adressiert oder vermieden werden. 
    Dafür werden in \fref{sec:prototype_sec} gesonderte Anforderungen erhoben.
    \medskip\\
    Der Auswahlprozess des verwendeten Frameworks wird in \fref{sec:prototype_framework} knapp dargestellt. 
    Nach einer kurzen Einführung in das Framework (\fref{sec:prototype_sec}), in der für den Prototypen relevante Eigenschaften, Funktionen und Komponenten erläutert werden, wird das Konzept des Prototypen und einige Implementierungsdetails vorgestellt (\fref{sec:prototype_arch}). 
