\subsection{Blockchain}
\label{sec:blockchain}
    Eine Blockchain ist eine immer größer werdene Kette von Einträgen, die dezentral gespeichert wird. 
    Erstmals 2008 von Satoshi Nakamoto, ursprünglich als Peer-to-Peer Electronic Cash System ,,Bitcoin`` erfunden, fand die Technologie aufgrund ihren vermeindlichen Vorteilen wie des dezentralen, anonymen Konzepts schnell viele <Anhänger>. \todo[color=cyan]{Einleitung schreiben}
    
    \subsubsection{Double-Spending Problem}
    \label{sec:blockchain_doublespend}
    Das Double Spending Problem beschreibt die Möglichkeit den selben digitalen Token als Käufer eines Guts mehrmals auszugeben\cite{Chohan2017}.
    Die Gefahrt besteht, da ein Token aus einer Datei besteht, die dupliziert und gefälscht werden kann\cite{Chohan2017}.
    Der Empfänger einer Transaktion kann die Echtheit und eventuell mit dem Token bereits getätigte Transaktionen im Normalfall nicht verifizieren\cite{Nakamoto2008}.
    Daher wird häufig eine \gls{ttp} zur Prüfung der Transaktion herangezogen.
    Somit muss der Token zur Überprüfung an die sogenannte Mint (primärer Produzent der Währung) zurückgegeben werden. 
    Dieser erzeugt einen neuen Token und gibt diesen an den Verkäufer weiter.
    Einzig diesem neuen Token kann vertraut werden, dass er nicht mehrfach genutzt wurde\cite{Nakamoto2008}.
    
    \subsubsection{Einführung in das Konzept}
    \label{sec:blockchain_introduction}
    Bitcoin ist die erste digitale Währung, die urprünglich ohne zentrale Authorität wie beispielsweise einer \gls{ttp} das bereits in \fref{sec:blockchain_doublespend} vorgestellte Double-Spending Problem lösen sollte\cite{Nakamoto2008}. 
    Übertragen auf Bitcoin: der Empfänger muss sichergehen können, dass die vorigen Besitzer keine vorherigen Transaktionen signiert haben \cite{Nakamoto2008}\todo[color=yellow]{Übersetzung okay?}.
    Dies wird mittels kryptographischem Beweis anstatt des laut \citeauthor{Nakamoto2008} vorstellten mangelhaften Modells der \gls{ttp} gelöst\todo[color=orange]{besser ausdrücken}.
    
    Das Konzept \citeauthor{Nakamoto2008}s ermöglichte es somit zwei Entitäten, die sich gegenseitig nicht vertrauen müssen eine direkte Transaktion druchzuführen.
    Es müssen also alle Teilnehmer in einem Bitcoin-Netzwerk alle Transaktionen innerhalb des Netzwerkes kennen (Analog: Rollen des Ausstellers bzw. des Mint übernehmen)\todo[color=orange]{nachlesen}.
    Ebenso müssen alle Transaktionen für alle Teilnehmer einsehbar sein, also veröffentlicht werden.
    Weiterhin benötigt das System eine einzige Historie, in welcher alle vergangenen Transaktionen in der Reihenfolge ihres Auftretens stehen.
    Zuletzt benötigt der Empfänger des Coins einen Beweis, dass sich die Mehrheit der Knonten zur Zeit der Transaktion einig waren, dass die aktuelle die erste entgegengenommene Transaktion des Coins ist. 
    \cite{Nakamoto2008}
    \todo[color=cyan]{timestamps?}
    \medskip\\
    
    \noindent Grundlegend ist die Blockchain eine verteilte Datenstruktur, die zwischen den Mitgliedern eines Netzwerkes repliziert und geteilt wird\cite{Christidis2016}.
    Die Knoten des Netzwerkes (Miner) fügen validierte, gegeseitig abgestimmte Transaktionen, die zu Blöcken gebündelt werden.
    Die Blockchain beinhaltet das maßgebliche ,,Hauptbuch'` von Transaktionen, welches im Endeffekt festlegt, wem was gehört.\todo[color=yellow]{meh}
    Das System ist so lange sicher, wie ,,ehrliche'` Knoten gemeinsam mehr Rechenleistung als eventuell zusammenarbeitende Angreifer haben\cite{Nakamoto2008}.
    
    Zunächst wurde mit digitalen Coins gehandelt, welche einer in Form einer Kette digitaler Signaturen modelliert wurden.
    In anderen Anwendungsbereichen außerhalb des Bereiches der Kryptowährungen wird von generalisierten Assets gesprochen.\todo[color=yellow]{richtig?}
    \begin{figure}[H]
        \missingfigure[figheight=4cm]{Bild mit Kette digitaler Signaturen}
    \end{figure}
    
    Assets werden von einem Versender zu einem Empfänger transferiert, in dem der Versender einen Hash der vorigen Transaktion und den öffentlichen Schlüssel des Empfängers mit seinem eigenen privaten Schlüssel digital signiert und diese Hash dann am Ende des Assets anfügt.
    Der Empfänger kann den Besitz des Assets über die Kette der digitalen Signaturen zurückverfolgen\cite{Nakamoto2008}.
    
    Getätigte Transaktionen sollen nicht rückgängig machbar sein, in dem die Rückrechnung des Beweises rechnerisch zu aufwändig sein soll\cite{Nakamoto2008}.
    Somit wird der Verkäufer vor Täuschung geschützt und Treuhandmechanismen können einfach implementiert werden\cite{Nakamoto2008}.  
    
    \begin{figure}[H]
        \missingfigure[figheight=4cm]{Bild mit Transaktion}
    \end{figure}
    
    
    \subsubsection{Sicherheit}
    \label{sec:blockchain_security}
    Byzantine Fault Tolerance
    
    \subsubsection{Identity Management}
    \label{sec:blockchain_identitymgmnt}
    \subsubsection{Self-Sovereign Identitity}
    \label{sec:blockchain_sovreign}

\subsection{Internet of Things}
    \subsubsection{Sicherheit}
    \subsubsection{Protokolle (BLE, Z-WAVE)}
    \subsubsection{Smart Homes}

\subsection{Smart Locks}
    \subsubsection{Häufig vorgefundene Gemeinsamkeiten}



