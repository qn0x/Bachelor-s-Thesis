\subsection{Bekannte Sicherheitslücken und Angriffe}
\label{sec:analysis_weaknesses}

	Schwachstellen und Sicherheitslücken in August Smart Lock nach \citeauthor{Ye2017}.
	Setzt u.U. voraus, dass der Angreifer Zugriff auf den mobilen Client des Benutzers hat und evtl. Jailbroken/rooted ist.
	In der Applikation werden nach den Untersuchungen von \citeauthor{Ye2017} ein Handshake-Key, Benutzeraccount und persönliche Informationen als Klartext innerhalb einer XML-Datei abgelegt.
	\begin{itemize}
		\item nutzt direktes Bluetooth-Pairing, welches nur auf dem konstanten/statischen Handshake-Key basiert
		\item keine Authentifizierung der Pairing-Partner zwischen Mobile Client und Lock
		\item Kopieren von der Einstellungen aus dem System von Alice.
		    Eve legt sich einen neuen Account an und kopiert die Einstellungen von Alice, sodass von ihrem Nutzerkonto auf diese zugegriffen wird. \textrightarrow Mandatory Access Control
		\item Personenbezogene Daten des Nutzers liegen in Klartext auf dem Dateisystem
		\item DOS: Wenn mehrere Nutzer gleichzeitig versuchen sich mit dem Smart Lock zu verbinden wird bei einem die App suspendiert, sodass niemand gleichzig Öffnen/Schließen-Befehle senden kann.
		    Somit kann ein Angreifer kontinuierlich wechseln Öffnen/Schließen-Befehle senden und der legitime Nutzer kann die Smart Lock nicht kontrollieren.
	   \item Angreifer repliziert die App und bringt den Nutzer dazu die bösartige App zu verwenden, da keine Authentifizierung zwischen App und Nutzer
	   \item Bluetooth interference
	   \item OS Level flaws
	\end{itemize}

	Andere Locks
	\begin{itemize}
		\item Kwikset Kevo smart lock but the physical lock contains serious vulnerabilities making it easily compromised, which only takes 10 seconds\cite{Ye2017}
		\item QuickLock Smart Lock does not encrypt the passwords and sends the password to the user who forgets the password in plaintext\cite{Ye2017}
		\item iBluLock Smart Lock 6-character password, which is vulnerable against the brute forcing attack\cite{Ye2017}
	\end{itemize}
	
	Paper von \citeauthor{Ho2016}:
	Zwei Angriffe werden vorgestellt und an fünf verschiedenen Modellen(August, Danalocl, Kevo, Okidokeys, Lockitron) getestet.
	An private Informationen kommen und unauthorisierten Zugang bekommen.
	\begin{itemize}
	    \item Systemdesign und Access Control Policies: Angreifer kann Mechanismen für den Entzug von Rechten und Zugriffsprotokollierungen umgehen, die von den meisten Geräten genutzt werden.
	    \item automatisches öffnen: wird oft ungewollt im beisein eines Angreifers geöffnet. 
	        Alle Systeme nutzen unsichere Mechanismen, um die vom Nutzer beabsichtigten Aktionen zu erfassen.
        \item Availability: Dadurch, dass das Smart Lock mit dem in \fref{fig:gateway} vorgestellen Kommunikationsmodell volkommen davon abängig ist, dass das Gerät des Owners die Nachrichten unangerührt weiterleitet, ist dieses Vertrauensmodell im Falle eines Entzugs von Rechten eines anderen Nutzers unpassend.
            Indem ein Angreifer, dessen Rechte entzogen werden soll jegliche Kommunikation zwischen Alice und dem remote Server des Herstellers verhindert.
            Dies kann beispielsweise am einfachsten mittels des Flugmodus erreicht werden.
            Gleichzeitig entgeht der Angreifer auch dem Access Logging.
            \todo[color=red]{Ho206, p.4, 3.3.3}

	\end{itemize}
	
	Threat Models
	\begin{itemize}
	    \item Physically-present Attacker: Angreifer beobachtet einen Nutzer bei der Interaktion mit dem Smart Lock und kann jederzeit damit interagieren, besitzt aber kein authorisiertes Gerät.
		\item Revoked Attacker: Besitzt legitimen Zugriff, der welcher aber zu einem spätereen Zeitpunkt wieder entzogen wird.
		\item Dieb: Der Angreifer stiehlt das authorisierte Gerät des Nutzers
		\item Relaying Attacker: Angreifer besitzt ein Bluetoothgerät, das mit anderen Bluetoothgeräten kommmunizieren kann, aber nicht für authorisiert ist.
	\end{itemize}
	
	Weitere im Paper von \citeauthor{Ye2017} aufgelistete Angriffe\todo[color=cyan]{sortieren und filtern}
	\begin{itemize}
	    \item August smart lock (earlier Version) has hard-coded secret key in the application source code \cite{Rose2016}
		\item August: does not perform the 2-factor authentication properly, and the hackers compromising the user email and text message could illegally control the lock
		\item does not perform the password reset process properly, and the	attackers can easily figure out the true verification code for resetting any passwords
		\item Bluetooth Jammer \textrightarrow DOS
		\item Mobile Device: Offizielle App fälschen oder bösartige App verwenden \textrightarrow Informationen des Nutzers stehlen \textrightarrow 
		\item Handshake Key Leakage Attack: in which the attacker is able to steal the handshake key from the smart lock, and	illegally and covertly control the lock using a third-party	device\\
		    Kann einfach durch ein vom Hersteller öffentliches Open Source Programm ausgenutzt werden ohne für den Besitzer sichtbare Spuren zu hinterlassen (und ohne die offizielle App zu benutzen)
		\item Owner Account Leakage Attack: in which the attacker is able to disguise himself/herself to be the owner, by logging into the lock owner’s account in the third-party device to control the smart lock without being discovered
		\item Personal Information Leakage Attack: in which the attacker is able to obtain the lock user information, which seriously threatens the user privacy; and
		\item Denial-of-service (DoS) Attack: in which the attacker disrupts the regular usage of smart lock, which dramatically	brings down the user experience
	\end{itemize}