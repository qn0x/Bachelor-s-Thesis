\subsection{Bekannte Sicherheitslücken und Angriffe}
\label{sec:analysis_weaknesses}

    Kategorisierung beispielsweise nach OWASP Top Ten oder Attack Surface Areas.

    In \fref{fig:sl_arch} lassen sich für eine Sicherheitsanalyse meherere Ansätze erkennen. 
    
    \subsubsection{}
    Schwachstellen und Sicherheitslücken in August Smart Lock nach \citeauthor{Ye2017}.
	Setzt u.U. voraus, dass der Angreifer Zugriff auf den mobilen Client des Benutzers hat und evtl. Jailbroken/rooted ist.
	In der Applikation werden nach den Untersuchungen von \citeauthor{Ye2017} ein Handshake-Key, Benutzeraccount und persönliche Informationen als Klartext innerhalb einer XML-Datei abgelegt.
	\begin{itemize}
		\item nutzt direktes Bluetooth-Pairing, welches nur auf dem konstanten/statischen Handshake-Key basiert
		\item keine Authentifizierung der Pairing-Partner zwischen Mobile Client und Lock
		\item Kopieren von der Einstellungen aus dem System von Alice.
		    Eve legt sich einen neuen Account an und kopiert die Einstellungen von Alice, sodass von ihrem Nutzerkonto auf diese zugegriffen wird. \textrightarrow Mandatory Access Control
		\item Personenbezogene Daten des Nutzers liegen in Klartext auf dem Dateisystem
		\item DOS: Wenn mehrere Nutzer gleichzeitig versuchen sich mit dem Smart Lock zu verbinden wird bei einem die App suspendiert, sodass niemand gleichzig Öffnen/Schließen-Befehle senden kann.
		    Somit kann ein Angreifer kontinuierlich wechseln Öffnen/Schließen-Befehle senden und der legitime Nutzer kann die Smart Lock nicht kontrollieren.
	   \item Angreifer repliziert die App und bringt den Nutzer dazu die bösartige App zu verwenden, da keine Authentifizierung zwischen App und Nutzer
	   \item Bluetooth interference
	   \item OS Level flaws
	\end{itemize}

	Andere Locks
	\begin{itemize}
		\item Kwikset Kevo smart lock but the physical lock contains serious vulnerabilities making it easily compromised, which only takes 10 seconds\cite{Ye2017}
		\item QuickLock Smart Lock does not encrypt the passwords and sends the password to the user who forgets the password in plaintext\cite{Ye2017}
		\item iBluLock Smart Lock 6-character password, which is vulnerable against the brute forcing attack\cite{Ye2017}
		\item Lockitron: direkte Verbindung des Schlosses mit den Servern des Herstellers \textrightarrow legitime Nutzer könnten ausgeschlossen werden, wenn dessen Server nicht erreichbar sind
	\end{itemize}
	
	Paper von \citeauthor{Ho2016}:
	Zwei Angriffe werden vorgestellt und an fünf verschiedenen Modellen(August, Danalock, Kevo, Okidokeys, Lockitron) getestet.
	An private Informationen kommen und unauthorisierten Zugang bekommen.
	\begin{itemize}
	    \item Systemdesign und Access Control Policies: Angreifer kann Mechanismen für den Entzug von Rechten und Zugriffsprotokollierungen umgehen, die von den meisten Geräten genutzt werden.
	    \item automatisches öffnen: wird oft ungewollt im beisein eines Angreifers geöffnet. 
	        Alle Systeme nutzen unsichere Mechanismen, um die vom Nutzer beabsichtigten Aktionen zu erfassen.
        \item State-Consistency Attacks: Dadurch, dass das Smart Lock mit dem in \fref{fig:gateway_arch} vorgestellen Kommunikationsmodell volkommen davon abängig ist, dass das Gerät des Owners die Nachrichten unangerührt weiterleitet, ist dieses Vertrauensmodell im Falle eines Entzugs von Rechten eines anderen Nutzers unpassend.
        
        \item Revocation Evasion: Indem ein Angreifer, dessen Rechte entzogen werden soll jegliche Kommunikation zwischen Alice und dem remote Server des Herstellers verhindert.
        Dies kann beispielsweise am einfachsten mittels des Flugmodus erreicht werden.
        Gleichzeitig entgeht der Angreifer auch dem Access Logging.
        Szenario mit Danalock: Alice möchte Mallory, einem Nutzer mit temprärem Zugang die Rechte entziehen. 
        Dazu nutzt sie die App, welche diese Information an den Server des Herstellers sendet und ihr sofort bestätigt, dass sie Rechte von Mallory nun entzogen wurden. 
        Dieser sendet eine Benachrichtigung darüber an Mallorys Gerät. 
        Befindet sich Mallorys Gerät zu diesem Zeitpunkt im Flugmodus, erreicht diese Benachrichtigung ihr Ziel nicht und das Schloss selbst bleibt unwissend über dieses Ereignis. 
        Selbst, wenn ein legitimes Gerät mit dem Schloss interagiert, wird die Benachrichtung über Mallorys Entzug der Recht nicht vom Server über jenes übertragen.
        Somit behält Mallory Zugang für das Schloss.
        \item Access Log Evasion: ist möglich, da, wie in dem oben beschriebenen Szenario davon ausgegangen wird, dass alle Geräte alle Aktionen an den remote Server senden.
        Szenario mit Danalock: Alice, Bob und Mallory haben einen legitimen Zugang.
        Bob ist Owner und kann die Access Logs einsehen.
        Mallory blockiert alle Pakete, die von der App an den remote Server gesendet werden während er mit dem Schloss via ÖFFNEN/SCHLIEßEN- Befehl interagiert.
        Somit werden seine Aktionen nicht mitgeschrieben, da das Schloss selbst diese Aktionen von Schloss selbst nicht zwischengespeichert und über einen anderen Client an den remote Server gesendet werden.
        Da diese Informationen auch nachdem Alice mit dem Schloss interagiert nicht über ihr Gerät an den Server weitergeleitet werden, kann Bob als Owner nicht sehen, dass Mallory das Schloss geöffenet oder geschlossen hat.
        \item Unwanted Unlocking (August und Danalock): Apps nutzen Location Services mittels Geofencing (50m Radius) auf dem Mobilgerät, über welche das Schloss, wenn in Reichweite von \gls{ble} und autorisiert, automatisch entriegelt wird.
        Ist Alice einmal innerhalb des Radiuses, wird das Schloss nicht automatisch wieder verschlossen so lange sie sich noch innerhalb der Reichweite des \gls{ble}-Moduls befindet - erst bei Austritt aus dem Radius wird das Schloss wieder verriegelt.
        Szenario: Das Schloss ist an Alices Haustür angebracht.
        Zusätzlich hat das Haus von Alice eine weitere Tür durch die ihr Haus betreten werden kann.
        Geht Alice nicht durch die Haustür, sondern durch die andere, bleibt die Haustür durch die oben Beschriebene Funktion entriegelt.
        Gleiches Problem entsteht, wenn Alice beispielsweise an beiden Türen Smart Locks angebracht hat.
        Bewegt sie sich mit ihrem Smartphone durch ihr Haus werden die Schlösser, sobald sie in \gls{ble}-Reichweite ist, automatisch entriegelt und nicht wieder automatisch verriegelt, da sie sich weiterhin innerhalb des Geofencing-Radius befindet.\\
        Kevo: Da dieses Schloss über einen Touch-To-Unlock-Mechanismus verfügt, funktioniert das oben Beschriebene Szenario nicht.
        Dieser Mechanismus erlaubt es Alice, wenn autorisiert und in \gls{ble}-Reichweite das Schloss mittels eines Fingertippens zu entriegeln.
        Jedoch ist es bei diesem Modell möglich, dass Alice sich innerhalb des Hauses befindet und ein Angreifer von außen mittels eines Fingertippens die Tür öffnen kann, da alle Bedingungen für ein legitimes Öffnen - minus der physischen Person - erfüllt sind.
        Allerdings ist dieser Angriff nur bei bestimmten Grundrissen möglich, da der Hersteller einen Algorithmus verwendet, der die Richtung, aus der das Bluetooth-Signal kommt, bestimmt und lediglich Signale entgegennimmt, die im 180$^{\circ}$-Winkel vor dem Schloss befinden.
        \item Relay Attacks (Kevo, Danalock, August): Ein Angreifer folgt Alice (außerhalb des Bluetooth- und Geofencingradius, aber in Bluetooth-Reichweite zu Alices Gerät) und überträgt das Signal an einen anderen Angreifer, der vor Alices Haus mit einem bluetoothfähigen Gerät wartet und mittels des übertragenen Signals nun alle Bedingungen erfüllt und das Schloss entriegeln kann.
        Bei Systemen mit Geofencing muss zusätzlich Alices Standort gespooft werden.
	\end{itemize}
	
	Threat Models
	\begin{itemize}
	    \item Physically-present Attacker: Angreifer beobachtet einen Nutzer bei der Interaktion mit dem Smart Lock und kann jederzeit damit interagieren, besitzt aber kein authorisiertes Gerät.
		\item Revoked Attacker: Besitzt legitimen Zugriff, der welcher aber zu einem spätereen Zeitpunkt wieder entzogen wird.
		\item Dieb: Der Angreifer stiehlt das authorisierte Gerät des Nutzers
		\item Relaying Attacker: Angreifer besitzt ein Bluetoothgerät, das mit anderen Bluetoothgeräten kommmunizieren kann, aber nicht für authorisiert ist.
	\end{itemize}
	
	Weitere im Paper von \citeauthor{Ye2017} aufgelistete Angriffe\todo[color=cyan]{sortieren und filtern}
	\begin{itemize}
	    \item August smart lock (earlier Version) has hard-coded secret key in the application source code \cite{Rose2016}
		\item August: does not perform the 2-factor authentication properly, and the hackers compromising the user email and text message could illegally control the lock
		\item does not perform the password reset process properly, and the	attackers can easily figure out the true verification code for resetting any passwords
		\item Bluetooth Jammer \textrightarrow DOS
		\item Mobile Device: Offizielle App fälschen oder bösartige App verwenden \textrightarrow Informationen des Nutzers stehlen \textrightarrow 
		\item Handshake Key Leakage Attack: in which the attacker is able to steal the handshake key from the smart lock, and	illegally and covertly control the lock using a third-party	device\\
		    Kann einfach durch ein vom Hersteller öffentliches Open Source Programm ausgenutzt werden ohne für den Besitzer sichtbare Spuren zu hinterlassen (und ohne die offizielle App zu benutzen)
		\item Owner Account Leakage Attack: in which the attacker is able to disguise himself/herself to be the owner, by logging into the lock owner’s account in the third-party device to control the smart lock without being discovered
		\item Personal Information Leakage Attack: in which the attacker is able to obtain the lock user information, which seriously threatens the user privacy; and
		\item Denial-of-service (DoS) Attack: in which the attacker disrupts the regular usage of smart lock, which dramatically brings down the user experience
	\end{itemize}
	
	Das in \fref{sec:sota_smart_locks} vorgestellte Schlüsselaustauschprotokoll lässt auf den ersten Blick folgende potentielle Angriffe zu\cite{Fuller2017}: 
	\begin{itemize}
	    \item Wenn der Angreifer an den Firmware Key gelangt, könnte dieser mit einer Chosen Plaintext-Attacke fabrizierte Nachrichten anstatt 64 random Bits den Server schicken  
	    \item der Session Key wird als Klartext vom Server an den Gast geschickt, sodass ein Mithörer an diesen gelangen könnte.
	        Er wird jedoch unverzüglich für die Kommunikation zwischen Schloss und Gast genutzt und direkt nach Nutzung wieder verworfen, sodass das Kennen des Session Keys für einen Angreifer laut \citeauthor{Fuller2017} eher unwahrscheinlich einen Nutzen hervorbringt.
	\end{itemize}
	
	Die in \fref{sec:sota_smart_locks} vorgestellte Kommunikation via Wifi zwischen dem Webserver und dem der Smartphone-App lässt auf den ersten Blick folgende potentielle Angriffe zu\cite{Fuller2017}\cite{Lariviere2015}: 
	\begin{itemize}
	    \item wenn sich ein Nutzer das erste Mal in der App anmeldet, wird eine zufällige ID generiert (installID), welche an den Webserver gesendet wird und für viele API-Aufrufe entweder als Teil des API-Keys oder zusätzlich zu diesem verwendet wird.
	    \item eine Zeit lang API Endpunkte authentifizieren den Nutzer nicht richtig und erlauben es einem Angreifer einen Gästezugang anzulegen, wenn dieser die \gls{uuid} kennt, welche, sollte man sich in der Reichweite des \gls{ble} befinden.
	    \item Ebenfalls möglich war eine Privilege Escalation, in dem ein String von 'user' auf 'superuser' geändert wurde.
	\end{itemize}
	
	Im Paper von \cite{Fuller2017} vorgestellte Angriffe:
	\begin{itemize}
	    \item ein Angreifer kappt die Internetverbindung, um einem Entzug der Rechte zu entgehen
	        \begin{itemize}
	            \item ein ein ehemaliger Nutzer mit der Rolle Owner kann seinen Zugriff behalten, indem er die Internetverbindung kappt, wie bei \cite{Ho2016}
	            \item ein Dieb umgeht das remote Logout-System
	                Wenn Auto-Unlock aktiviert ist, er aber Das Gerät nicht entsperren kann - es ist aber möglich via August "Lost device", kann aber umgangen werden, indem der Flugmodus aktiviert wird
	                Ist aber weniger schlimm, ähnlich einem verlorenen Schlüssel
	        \end{itemize}
	    \item ein Angreifer mit einem in der Zukunft auf Zeit limitierten Gastzugang ändert die Zeiteinstellung auf seinem Gerät, um zu einem anderen Zeitpunkt Zugriff zu erhalten
	    \item Bluetooth-Verkehr zwischen dem Smartphone und dem Schloss mitlesen
	    \item andere Möglichkeiten
	\end{itemize}
	
	Von \cite{Rose2016} vorgestellt:
	\begin{itemize}
	    \item Plaint Text Password (Quicklock, iBluLock, Pantraco Phantomlock): 
	        z.B. einfach Opcode (Herstellserspezifisch), aktuelles Passwort, neues Passwort konkatiniert und gesendet. 
	        Wird dies gesendet, dann wird der Owner mit einem neuen Passwort ausgesperrt und ein hard Reset ist nötig. 
	        Jedoch ist diser Angriff nur möglich, wenn das Schloss in dem Moment, in dem die Nachricht gesendet wird, entriegelt ist.
	    \item Replay-Angriff (Ceomate Bluetooth, Elecycle Smart Padlack, Vians Bluetooth Smart Doorlock, Lagute Sciener Smart Doorlock)
	    \item Fuzzing (Okidokey Smart Doorlock): 
	        Indem man einige Bytes eines gültigen Befehls ändert und versucht das Schloss in einen Fehlerzustand zu bringen. 
	        Okidokey:
	        \begin{itemize}
	            \item über Sniffing an einen gültigen Befehl kommen, bei dem ein scheinbar einzigartiger Schlüssel gesendet wird
	            \item indem man das dritte Byte zu 0x00 ändert, bringt man das Schloss in einen Fehlerzustand, in welchem sich das Schloss automatisch entriegelt
	            9348(Opcode?)|b6cad7299ec1481791303d7c90d549352398(Key)
	            \item während dieses Fehlerzustands ist das Schloss für den Nutzer unbrauchbar
	        \end{itemize}
	    \item Decompiling APKs (Poly-Control Danalock Doorlock): 
	        \begin{itemize}
	            \item APKs vom Android-Gerät holen
	            \item .dex zu .jar konvertieren
	            \item .jar dekompilieren
	            \item zeigt Verschlüsselungsmethode und hardcoded Secret (Kerckoff's Prinzip)
	            \item XOR(passwort, secret) \textrightarrow es ist also einfach an das Passwort zu kommen \textrightarrow Entschlüsseln der Kommunikation ist möglich
	        \end{itemize}
	    \item Device Spoofing (Mesh Motion Bitlock Padlock): 
	        \begin{itemize}
	            \item Angreifer gibt sich als Schloss aus, um das Passwort des Nutzers zu stehlen
	            \item möglich, da ein vorhersehbarer Nonce geschickt wird und die App beim Nutzer im Hintergrund läuft und Befehle ohne Nutzerinteraktion sendet.
	        \end{itemize}
	    \item Brute-Forcing: bei Quicklock möglich mit 8-Ziffern PIN (100.000.000 Kombinationen)
	\end{itemize}
	