\section{Ergebnis}
\label{sec:end}
    Abwägungen zum Einsatz der Blockchain-Technologie im \gls{iot}
    \begin{itemize}[noitemsep]
        \item Blockchain tut sehr viel in Richtung Identity Management, nützt aber nichts, wenn das Umfeld (Übertragung via HTTP und nicht HTTPS, keine 2-Factor Authentication, etc.) nicht gesichert ist.
        \item kurz Vorteile/Nachteile von Blockchain
        \item Manche Anforderungen aus \fref{sec:prototype_requirements} nicht direkt umsetzbar.
            Einige dieser sind dennoch essentiell für die Sicherheit einer \gls{iot}-Applikation.
        \item Grenzen der Aussagekraft des Prototypen
		    \begin{itemize}[noitemsep]
		        \item Mindestanzahl an Teilnehmern im Blockchain-Netzwerk wegen \gls{bft} \textrightarrow andere Konsensalgorithmen bzw. -mechanismen
		        \item Privatsphäre in private Blockchains
		        \item Passwortrücksetzung
		        \item falsche Sicherheitskonfigurationen auf Host des Prototypen kann nicht betrachtet werden
		    \end{itemize}
		\item Worüber keine Aussagen getroffen werden konnten, z.B. Automatisches Öffnen-/\-Schließen konnte gar nicht getestet werden
    \end{itemize}
    Daher lohnt es sich nach der Untersuchung (oder lohnt es sich nicht) innerhalb des Rahmens dieser Arbeit die Blockchain-Technologie im \gls{iot}, spezifisch bei Smart Locks einzusetzen.
    
    
\section{Ausblick}
\label{sec:end_further}
	\begin{itemize}[noitemsep]
		\item Auf die Konzepte Self-Sovereign Identitiy, Decentralized Identifiers eingehen und referenzieren
		\item Ideen zur sicherheitstechnischen Erweiterung des Prototypen erläutern
		\begin{itemize}[noitemsep]
		    \item Event, wenn der Schlüssel häufiger als X mal pro Stunde oder Tag benutzt werden
		    \item Granularere Verteilung der Rollen im Netzwerk, auf einzelne Locks jeweils Owner und Guest Rollen
		    \item Security events generieren, welche von den jeweiligen Ownern eingesehen werden können und somit alarmier werden
		    \item Inputvalidierungen bei Nutzereingaben
		    \item Skalierung? Wenn der Hersteller für viele Kunden jeweils einen eigenen Knoten betreiben muss
		\end{itemize}
	\end{itemize}
