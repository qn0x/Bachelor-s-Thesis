\subsection{Anforderungen an den Prototypen}
\label{sec:prototype_requirements}
    Nichtfunktionale Anforderungen werden in dieser Arbeit nicht betrachtet. 
    Die hier aufgelisteten funktionalen Anforderungen, sowie die Sicherheitsanforderungen, die als umsetzbar aufgeführt werden, werden vollständig im Prototypen umgesetzt. 
    
\subsubsection{Funktionale Anfoderungen}
\label{sec:prototype_func_req}
    Aus dem Beispiel des August Smart Lock, das unter anderen in \fref{sec:sota_smart_locks} vorgestellt wurden, ergeben sich folgende funktionale Anforderungen (FA) als minimale Basisfunktionen:
    \begin{FA}[noitemsep]
        \item \label{fa:1}Nutzer verwalten: Hinzufügen von neuen Nutzern
        \item \label{fa:2}Nutzer verwalten: Entfernen von Nutzern
        \item \label{fa:3}Schloss öffnen
        \item \label{fa:4}Schloss schließen
        \item \label{fa:5}Vergabe der Berechtigung für die Funktion ,,Schloss öffnen``
        \item \label{fa:6}Entzug der Berechtigung für die Funktion ,,Schloss öffnen``
        \item \label{fa:7}Vergabe der Berechtigung für die Funktionen unter ,,Nutzer verwalten``
        \item \label{fa:8}Entzug der Berechtigung für die Funktionen unter ,,Nutzer verwalten``
        \item \label{fa:9}Die Nutzung eines Schlüssels kann optional zeitlich eingeschränkt werden können
        \item \label{fa:10}Ein Schloss soll zurückgesetzt werden können
        \item \label{fa:11}Ein neues Schloss in das Netzwerk aufgenommen werden können
        \item \label{fa:12}Das Netzwerk soll durch einen Hersteller initialisiert werden können
    \end{FA}

\subsubsection{Sicherheitsanforderungen}
\label{sec:prototype_sec}
    Zusätzlich zu den funktionalen Anforderungen für den Prototypen, lassen sich aus den in \fref{sec:analysis_cvss} gefundenen Schwachstellen und Sicherheitslücken einige Sicherheitsnforderungen (SA) ableiten, welche mit deren Umsetzung dabei helfen diese zu vermeiden.
    Die abgeleiteten Anforderungen werden nach der Möglichkeit und Unmöglichkeit der potentiellen Umsetzung mittels einer Blockchain gruppiert.
    Sollten die Anforderungen direkt auf die vermeidung einzelner vorgestellter Schwachstellen abzielen wird auf diese refernziert.
    \bigskip\\
    \noindent \textbf{Umsetzbare Sicherheitsanforderungen}:
    \begin{SA}[noitemsep]
        \item \label{sa:1}Gegenseitige Authentifizierung des Nutzers und des Smart Locks, speziell auch das Sicherstellen, dass die empfangenen Daten von der Richtigen Quelle kommen, um sicherzustellen, dass niemand anderes den zwischen beiden Parteien steht und Nachrichten manipulieren kann (Data Origin Authentication, Entity Authentication)\\
            \textrightarrow\ \hyperref[vuln:phishing]{Phishing}, \hyperref[vuln:replay]{Replay}, \hyperref[vuln:spoofing]{Device Spoofing}, 
        \item \label{sa:2}Mandatory Role Based Access Control für die Systeme, Daten auf den Systemen und Funktionen innerhalb eines Systems (Authorization) \cite{Miessler2015}\\ 
            \textrightarrow\ \hyperref[vuln:userdata]{Extracted User Data}, \hyperref[vuln:usersettings]{Stolen User Settings}
        \item \label{sa:3}Vollständiges und wohldefiniertes Protokoll bei der Kommunikation von Teilnehmern des Netzwerks\\
            \textrightarrow\ \hyperref[vuln:fuzzing]{Fuzzing}
        \item \label{sa:4}Gegen Angriffe, die eine Diskrepanz des aktuellen Zustands Systems zwischen verschiedenen Teilnehmer und Knoten im Netzwerk ausnutzen\\
            \textrightarrow ,,eventual consistency`` (Availability, Non-Repudiation)\cite{Ho2016}
        \item \label{sa:5}nicht-authorisierten Zugriff verhindern bzw. nur dann öffnen/schließen, wenn der Nutzer, der den Befehl gibt, authorisiert ist(Whitelisting) \cite{Ho2016}
        \item \label{sa:6}Access Logging \& Integrität der Logs gewährleisten\\
            \textrightarrow\ \hyperref[vuln:accesslogevasion]{Access Log Evasion}
        \item \label{sa:7}Timestamping\\
            \textrightarrow\ \hyperref[vuln:replay]{Replay}, \hyperref[vuln:revocationevasion]{Revocation Evasion}, \hyperref[vuln:accesslogevasion]{Access Log Evasion}
        \item \label{sa:8}Consistency, Availability, Partition Tolerance (CAP-Theorem \cite{Brewer2012})\\
            \textrightarrow\ \hyperref[vuln:accesslogevasion]{Access Log Evasion}
        \item \label{sa:9}sicheres Kommunikationsprotokoll für jegliche Übertragung von Nachrichten\cite{Miessler,Ye2017}
        \item \label{sa:10}keine statischen Passwörter speichern oder übertragen\cite{Rose2016}
    \end{SA}
    \medskip
    \noindent \textbf{Nicht umsetzbare Sicherheitsanforderungen}:
    \begin{itemize}[noitemsep]
        \item Sicherung der (relevanten) Daten, bevor diese auf dem Mobilgerät gespeichert werden\cite{Ye2017}\\
            \textrightarrow\ für den Prototypen ist keine mobile Komponente vorgesehen
        \item 2-Factor Authentication, falls möglich\cite{Rose2016,Miessler2015}\\
            \textrightarrow\ der Prototyp arbeitet nur mit der generierten \gls{api}, der Authentifizierungsmechanismus wird vom Framework voregegeben?
        \item Bei Verwendung eines Nonce-Wertes: echt zufälliger Nonce-Wert mit 8-16 Bytes\cite{Rose2016}\\
            \textrightarrow\ s.o.
        \item ordentlich implementierte AES-Verschlüsselung\cite{Rose2016}\\
            \textrightarrow\ s.o.
        \item gegen Relay-Attacken: Distance-bounding Protocols, die via RTT die Distanz des Nutzers zum Schloss bestimmen\cite{Ho2016}\\ 
            \textrightarrow\ keine mobile Komponente
        \item Web Interface: keine Default-Einstellungen, Passwortrücksetzungen sichern (keine Metainformationen herausgeben, z.B. Account vorhanden/\-nicht vorhanden, keine XSS, SQLi oder CSRF, Anmeldedaten nicht im Netzwerkverkehr offenlegen, keine schwachen Passwörter zulassen, Lockout nach 3-5 Versuchen (kein Brute-Force)\cite{Miessler2015}\\
            \textrightarrow\ kein Webinterface für Nutzer des Prototypen
    \end{itemize}
    