\section*{Fragen}
    \begin{itemize}
        \item Soll ich die Quellen mit abgeben, weil man nicht so einfach an alle rankommt? Wenn ja, wie?\\
            Könnte als Archiv oder als bib mit Verlinkungen auf die Quellen im Dateisystem sein
        \item Soll auch in irgendeiner Form eine Anleitung zur Nutzung des Prototypen dabei sein?
        \item Methodik des Vorgehens bei Kapitel oder nach Problemstellung in der Einleitung?
        \item Erklärung des \gls{cvss} bei den Grundlagen oder erst bei der Analyse?
        \item Berechnungsformeln des \gls{cvss} als Listings lassen oder in Equations umwandeln?
    \end{itemize}
    
\section*{How to todos:}
    \begin{itemize}
        \item \textcolor{cyan}{fehlender Inhalt}
        \item \textcolor{yellow}{Unsicher}
        \item \textcolor{orange}{besser ausdrücken}
        \item \textcolor{red}{falsch}
        \item \textcolor{green}{Korrekturlesen}
    \end{itemize}
    
\begin{table}[H]
\centering
\begin{tabular}{l|c|c|c}
                               & \textbf{Seitenanzahl (geplant)} & \textbf{Seitenanzahl (ist)} & \textbf{Status}              \\ \hline
\textbf{Einleitung} & 2 - 3                           & {\color[HTML]{F8FF00} 1}    & {\color[HTML]{34FF34} Done}  \\ \hline
\textbf{Problemstellung}       & 1 - 2                           & {\color[HTML]{34FF34} 2}    & {\color[HTML]{34FF34} Done}  \\ \hline
\textbf{Grundlagen}            & 10 - 15                         & {\color[HTML]{F8FF00} 14}   & {\color[HTML]{F8FF00} Doing} \\ \hline
\textbf{Analyse}               & 5 - 10                          & {\color[HTML]{F8FF00} 8}    & {\color[HTML]{F8FF00} Doing} \\ \hline
\textbf{Prototyp}              & 15                              & {\color[HTML]{FE0000} 1}    & {\color[HTML]{FE0000} Todo}  \\ \hline
\textbf{Evaluation}            & 5 - 10                          & {\color[HTML]{FE0000} 1}    & {\color[HTML]{FE0000} Todo}  \\ \hline
\textbf{Vergleich}             & 3                               & {\color[HTML]{FE0000} 1}    & {\color[HTML]{FE0000} Todo}  \\ \hline
\textbf{Schluss}               & 2 - 3                           & {\color[HTML]{FE0000} 1}    & {\color[HTML]{FE0000} Todo}  \\
\end{tabular}
\end{table}