\section{Source Code des Prototypen}
\label{apx:prototype}
    Der Source Code des Prototypen, sowie eine Anleitung zur Nutzung sind im entsprechenden  \href{https://github.com/qn0x/smart-lock-prototype}{Repository} zu finden.
    Zudem sind im Folgedenden die wichtigsten Bestandteile des Prototypen angehängt.
    \vspace{-1em}
    \subsection{Modelle}
        \begin{lstlisting}[caption={Assets},label=model_assets,captionpos=b]
asset LockKey identified by keyId {
    o String keyId
    o DateTime expirationDate optional
    --> Lock lock
    --> User owner
    --> User issuer
}
        \end{lstlisting}
        
        \begin{lstlisting}[caption={Participant-Typen},label=model_participants,captionpos=b]
participant User identified by userId {
    o String userId
    o String firstName
    o String lastName
    o UserRole role  // "OWNER" oder "USER"
    --> LockKey[] keys optional
}

participant Lock identified by lockId {
    o String lockId
    o String name
    o LockState state  // "LOCKED" oder "UNLOCKED"
    --> LockKey[] keyInUse
    --> Vendor vendor
}

participant Vendor identified by vendorId {
    o String vendorId
    --> Lock[] locks
}
        \end{lstlisting}
        \vspace{1em}
        \begin{lstlisting}[caption={Auswahl relevanter Transactions},label=model_transactopms,captionpos=b]
transaction Unlock {
    --> Lock lock
}

transaction CloseLock {
    --> Lock lock
}

transaction GrantUnlock {
    --> Lock lock
    o DateTime expirationDate optional
    --> User grantee
}

transaction RevokeUnlock {
    --> Lock lock
    --> User revokee
}

transaction GrantOwner {
    --> User grantee
}

transaction RevokeOwner {
    --> User revokee
}

transaction RemoveLock {
    --> Lock lock
}

transaction RegisterLock {
    --> Lock lock
}
        \end{lstlisting}
    \newpage
    \subsection{Access Control}
    \begin{lstlisting}[caption={Beispiel der Access Control Rules für ein Schloss},label=acl1,captionpos=b]
rule OwnersCanUpdateTheLock {
    description: "owners can create, read, update and delete the lock"
    participant(u): "de.hftl.user.User"
    operation: ALL
    resource(l): "de.hftl.lock.Lock"
    condition: (u.role == "OWNER")
    action: ALLOW
}

rule AnyoneCanSeeTheLocks {
    description: "anyone can see all locks"
    participant: "de.hftl.**"
    operation: READ
    resource: "de.hftl.lock.Lock"
    action: ALLOW
}

rule KeyHoldersCanUpdateTheLockUnlock {
    description: "key holders can update the lock while unlocking"
    participant(u): "de.hftl.user.User"
    operation: UPDATE
    resource(l): "de.hftl.lock.Lock"
    transaction(tx): "de.hftl.lock.Unlock"
    condition: (
      tx.lockKey.lock.getIdentifier() == tx.lock.getIdentifier()
        && tx.lock.getIdentifier() == l.getIdentifier()
        && l.state == "LOCKED"
        && u.keys.some(function (lockKey) {
          return 
             lockKey.lock.getIdentifier() === l.getIdentifier();
        })
    )
    action: ALLOW
}
\end{lstlisting}
\newpage
\begin{lstlisting}[caption={Beispiel der Access Control Rules für ein Schloss (fortgesetzt)},label=acl2,captionpos=b]
rule AnyUserCanCloseTheLock {
    description: "Any user can close the lock if it is open"
    participant: "de.hftl.user.User"
    operation: UPDATE
    resource(l): "de.hftl.lock.Lock"
    transaction(tx): "de.hftl.lock.CloseLock"
    condition: (
        tx.lock.getIdentifier() == l.getIdentifier()
          && l.state == "UNLOCKED"
    )
    action: ALLOW
}

rule DenyLockModifications {
    description: "no one can modify the lock"
    participant: "de.hftl.**"
    operation: ALL
    resource: "de.hftl.lock.Lock"
    action: DENY
}
    \end{lstlisting}
    
    \newpage
    \subsection{Transaktionslogik}
\begin{lstlisting}[
caption={Verarbeitung der \colorbox{light-gray}{\lstinline{CloseLock}} Transaktion},
label=tx1
,captionpos=b
,numbers=left
,language=JavaScript
,numbersep=-1mm
]
  function closeLock(tx) {
	var lock = tx.lock;
	
	// check if lock is open first
	if (lock.state == "UNLOCKED") {
		lock.keyInUse = new Array();
		lock.state = "LOCKED";
		return getAssetRegistry('de.hftl.lock.Lock')
		.then(function (assetRegistry) {
			return assetRegistry.update(lock);
  		});
  	}
  }
\end{lstlisting}

\begin{lstlisting}[
caption={Verarbeitung der \colorbox{light-gray}{\lstinline{InitializeNetwork}} Transaktion},
label=tx2
,captionpos=b
,numbers=left
,language=JavaScript
,numbersep=-1mm
]
  function initializeNetwork(tx) {
	var id = generateID();
	var factory = getFactory();
	var firstUser = factory.newResource('de.hftl.user', 'User', id);
	firstUser.firstName = tx.firstName;
	firstUser.lastName = tx.lastName;
	firstUser.role = "OWNER";
	
	return getParticipantRegistry('de.hftl.user.User')
	.then(function (userRegistry) {
		return userRegistry.add(firstUser);
	});
  }
\end{lstlisting}
\newpage
\begin{lstlisting}[
caption={Verarbeitung der \colorbox{light-gray}{\lstinline{GrantKey}} Transaktion},
label=tx3
,captionpos=b
,numbers=left
,language=JavaScript
,numbersep=-1mm
]
  function grantKey(tx) {
	// generate a random id
	let id = generateID();
	
	// create a new key
	var factory = getFactory();
	var key = factory.newResource('de.hftl.lock', 'LockKey', id);
	if (tx.expirationDate !== null) {
		key.expirationDate = tx.expirationDate;
	}
	key.lock = tx.lock;
	key.owner = tx.grantee;
	key.issuer = getCurrentParticipant();
	
	let grantee = tx.grantee;
	// check if grantee has no keys yet
	if (grantee.keys == null) {
		grantee.keys = new Array();
	}
	grantee.keys.push(key);
	
	// add the new key to the key asset registry.
	return getAssetRegistry('de.hftl.lock.LockKey')
	.then(function (lockKeyRegistry) {
		return lockKeyRegistry.add(key);
	})
	.then(function () {
		return updateUserInRegistry(grantee);
	});
  }
\end{lstlisting}