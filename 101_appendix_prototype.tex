\section{Source Code des Prototypen}
\label{apx:prototype}
    Der Source Code des Prototypen, sowie eine Anleitung zur Nutzung sind unter \sloppy\url{https://github.com/qn0x/smart-lock-prototype} zu finden.
    Zudem sind im Folgedenden die wichtigsten Bestandteile des Prototypen angehängt.
    
    \subsection{Modelle}
        \begin{lstlisting}[caption={Assets},label=model_assets,captionpos=b]
asset LockKey identified by keyId {
    o String keyId
    o DateTime expirationDate
    o DateTime lastUsed
    --> Lock lockId
    --> User owner
    --> User issuer
}
        \end{lstlisting}
        \vspace{1em}
        \begin{lstlisting}[caption={Participants},label=model_participants,captionpos=b]
participant User identified by userId {
    o String userId
    o String firstName
    o String lastName
    o UserRole role
}

participant Lock identified by lockId {
    o String lockId
    o String name
    o DateTime lastUnlocked
    o LockState state
}

enum UserRole {
  o USER
  o ADMIN
}

enum LockState {
  o UNLOCKED
  o LOCKED
}
        \end{lstlisting}
        \vspace{1em}
        \begin{lstlisting}[caption={Transactions},label=model_transactopms,captionpos=b]
transaction Unlock {
    --> LockKey lockKey
    --> Lock lock
}

transaction GrantUnlock {
    --> LockKey lockKey
    --> Lock lock
    --> User issuer
    --> User grantee
}

transaction RevokeUnlock {
    --> LockKey lockKey
    --> Lock lock
    --> User issuer
    --> User revokee
}

transaction GrantAdmin {
    --> User issuer
    --> User grantee
}

transaction RevokeAdmin {
    --> User issuer
    --> User revokee
}
        \end{lstlisting}
    
    \subsection{Access Control}
    \subsection{Transaktionslogik}
\newpage
\section{Verwendung des Prototypen}
    For Dummies. 
