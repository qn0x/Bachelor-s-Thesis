\section{Vergleich des Prototypen mit den Analyseergebnissen}
\label{sec:comparison}
    Als Grundlage für den Vergleich des Prototypen mit den Analyseergebnissen dienen die Tabellen \ref{tab:vulns_cvss_short} und \ref{tab:eval_cvss_short}.
    Da sich die Problemstellung auf die Verwendung einer Blockchain-Technologie im Anwendungsfall eines Smart Locks bezieht, werden zunächst von den Analyseergebnissen die Schwachstellen gefiltert, die prinzipiell nicht durch deren Verwendung behoben werden können. 
    Dazu gehören beispielsweise Schwachstellen, die die Übertragung per Bluetooth betreffen, sowie Schwachstellen der Mobilen- und Webkomponenten, als auch die physische Sicherheit. 
    \medskip\\
    Auch bei der Evaluation des Prototypen befinden sich unter den gefundenen Schwachstellen einige, die nicht aufgrund einer Blockchain bestehen. 
    \medskip\\
    Nachdem alle irrelevanten Schwachstellen aus dem Vergleich genommen worden sind, bleibel Folgende:
    \missingfigure{2 Tabellen mit den übrig gebliebenen Schwachstellen}
    \bigskip\\
    Allgemeiner Vergleich der Scores
    \medskip\\
    Vergleich anhand er Kategorien
    \bigskip\\
    Die Probleme der Proxy-Architektur wären damit beseitigt, aber jedoch bleiben die Probleme der direkten Architektur erhalten.
    