\section{Vergleich des Prototypen mit den Analyseergebnissen}
\label{sec:comparison}
    Da die Arbeit sich um die Verwendung einer Blockchain-Technologie dreht, werden zunächst alle Schwachstellen gefiltert, die prinzipiell nicht durch deren Verwendung behoben werden können.
    Dazu gehören beispielsweise die Übertragung per Bluetooth (unsichere Pairing-Mechanismen und Session Keys), sowie physische Sicherheit und blabla\todo[color=cyan]{weitere auflisten}

%%%%%%%%%%%%%%%%%%%%%%%%%%%%%%
    Faktischer Vergleich beider Ergebnisse anhand der Kategorien, der Scores der Schwachstellen, Aufwand.
    \begin{itemize}[noitemsep]
        \item evtl. noch eine Gruppierung, der in \fref{sec:evaluation} gefunden Schwachstellen solche, die mit Erweiterung des Prototypen mitigiert werden können und jene, die nicht wegzubekommen sind?
        \item Nicht alle in \fref{sec:evaluation} gefundenen potentiellen Schwachstellen sind auf eine Blockchain zurückzuführen.
        \item Die Probleme der Proxy-Architektur wären damit beseitigt, aber jedoch bleiben die Probleme der direkten Architektur erhalten
    \end{itemize}
    
    