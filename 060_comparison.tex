\section{Vergleich des Prototypen mit den Analyseergebnissen}
\label{sec:comparison}
    Als Grundlage für den Vergleich des Prototypen mit den Analyseergebnissen dienen die Tabellen \ref{tab:vulns_cvss_short} und \ref{tab:eval_cvss_short}.
    Da sich die Problemstellung um die Verwendung einer Blockchain-Technologie dreht, werden zunächst von den Analyseergebnissen die Schwachstellen gefiltert, die prinzipiell nicht durch deren Verwendung behoben werden können. 
    Dazu gehören beispielsweise Schwachstellen, die die Übertragung per Bluetooth betreffen, sowie Schwachstellen der Mobilen- und Webkomponenten der Smart Lock-Systeme ,als auch die physische Sicherheit. 
    \medskip\\
    Auch bei der Evaluation des Prototypen sind einige Schwachstellen dabei, die nicht durch die Blpckchain entstanden sind. 
    \bigskip\\
    Allgemeiner Vergleich der Scores
    \medskip\\
    Vergleich anhand er Kategorien
    \bigskip\\
    Die Probleme der Proxy-Architektur wären damit beseitigt, aber jedoch bleiben die Probleme der direkten Architektur erhalten.
    