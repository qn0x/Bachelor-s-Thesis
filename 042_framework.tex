\subsection{Auswahl des Frameworks}
\label{sec:prototype_framework}
    Da die die Wahl des Frameworks eine wesentliche Entscheidung für die Arbeit und somit auch essentiell für die Bearbeitung der Problemstellung darstellt, wird diese knapp erläutert.
    \medskip\\
    Es existieren diverse Frameworks mit verschiedenen Anwendungsgebieten, daher sollte das letztendlich verwendete Framework folgenden Kriterien entsprechen:
    \begin{itemize}[noitemsep]
        \item Open Source \textrightarrow\ eigene Einflussmöglichkeit
        \item aktive Community \textrightarrow\ Unterstützung bei Schwierigkeiten während der Entwicklung
        \item aktive Weiterentwicklung \textrightarrow\ es besteht die Möglichkeit, dass das Framework auch in Produkten auf dem Markt genutzt werden könnte
        \item Auswahl an Sicherheitsmechanismen \textrightarrow\ Evaluierung von verschiedenen Konfigurationen möglich
    \end{itemize}
    \medskip
    \textbf{Auswahl verfügbarer Frameworks nach erster Recherche}
    \begin{itemize}[noitemsep]
        \item Hyperledger (Auswahl an verschiedenen Projekten mit unteschiedlichen Einsatzszenarien):
            \begin{itemize}[noitemsep]
                \item Iroha: einfaches Framework, das auch auf mobilen Geräten ist, Automatisierung über Kubernetes, hat jedoch noch kein major Release
                \item Fabric: Bekanntestes und am weitesten entwickeltes Framework, sehr viele verschiedene Features, auf Enterpriseanwendung ausgelegt, viele Konfigurationsmöglichkeiten, komplex
                \item Composer: Abstrahierung des Fabric Frameworks, für schnelle Entwicklung geeignet, explizite rollenbasierte Zugriffskontrolle
            \end{itemize}
        \item Exonum: private permissioned Blockchains, enthält einen Client, erweiterbar, noch ohne major Release, kein secure Storage für private Keys auf einem Knoten
        \item Quorum: basiert auf Ethereum, permissioned
    \end{itemize}
    Die Auswahl des Frameworks für den Prototypen fiel auf Hyperledger Composer.
