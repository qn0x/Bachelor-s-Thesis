\subsection{Auswahl des Frameworks}
\label{sec:prototype_framework}
    Da die die Wahl des Frameworks eine wesentliche Entscheidung für die Arbeit und somit auch essentiell für die Bearbeitung der Problemstellung darstellt, wird diese knapp erläutert.
    \medskip\\
    Es existieren diverse Frameworks mit unterschiedlichem Fokus auf verschiedene Anwendungsgebiete, daher soll das letztendlich verwendete Framework folgenden Kriterien entsprechen:
    \begin{itemize}[noitemsep]
        \item Open Source \textrightarrow\ eigene Einflussmöglichkeit
        \item aktive Community \textrightarrow\ Unterstützung bei Schwierigkeiten während der Entwicklung des Prototypen
        \item aktive Weiterentwicklung \textrightarrow\ es besteht die Möglichkeit, dass das Framework auch in zukünftigen Produkten auf dem Markt genutzt werden könnte
        \item Auswahl an Sicherheitsmechanismen \textrightarrow\ Evaluierung von verschiedenen Konfigurationen ist möglich
        \item ausführliche Dokumentation \textrightarrow\ Da sich die Implementierungen des Blockchain-Konzepts stark unterscheiden ist es dringend nötig, dass eine gute, ausführliche Dokumentation möglichst detailliert die verwendeten Mechanismen beschreibt.
        \item Unterstützung von privaten Blockchains
    \end{itemize}
    \medskip
    \textbf{Auswahl verfügbarer Frameworks nach erster Recherche}
    \begin{itemize}[noitemsep]
        \item Hyperledger (verschiedene Projekte mit unteschiedlichen Einsatzszenarien):
            \begin{itemize}[noitemsep]
                \item Iroha: einfaches Framework, das auch auf mobilen Geräten ist, Automatisierung über Kubernetes, hat jedoch noch kein major Release
                \item Fabric: Bekanntestes und am weitesten entwickeltes Framework, sehr viele verschiedene Features, auf Enterpriseanwendung ausgelegt, viele Konfigurationsmöglichkeiten, komplex
                \item Composer: Abstrahierung des Fabric Frameworks, für schnelle Entwicklung geeignet, explizite rollenbasierte Zugriffskontrolle und eigene Modellierungssprache für Teilnehmer, Transaktionen und Assets
            \end{itemize}
        \item Exonum: private permissioned Blockchains, enthält einen Client, erweiterbar, noch ohne major Release, kein secure Storage für private Keys auf einem Knoten
    \end{itemize}
    Die Auswahl des Frameworks für den Prototypen fiel auf Hyperledger Composer, da fast alle Frameworks, mit Ausnahme von Fabric, noch in einem sehr unreifen Entwicklungsstadium sind.
    Fabric selbst ist für die Entwicklung eines einfachen Prototypen zu komplex.
