\subsection{Architektur und Funktionsweise}
\label{sec:prototype_arch} 
    Da momentan weder eine Standardisierung noch Best Practices von Blockchainarchitekturen existieren, wurde die Architektur des Prototypen anhand der Dokumentation des Frameworks\cite{ComposerDocs}  ausgearbeitet.
    \medskip\\
    Notizen für den Prototypen:
    \begin{itemize}[noitemsep]
        \item Ein Asset als Token für (,,du darfst die Tür aufmachen``) je berechtigtem Nutzer \textrightarrow\ zurückverfolgbar (Accountability)
        \item Framework bietet in irgendeiner Form Access Control an?
        \item How to ensure a consistent state across all entities?...
        \item keine Anonymisierung, da der Access Log auditable sein muss
        \item Hyperledger bietet CA für Identitäten, Authentifizierung \textrightarrow\ wie ist das bei Composer?
        \item \begin{itemize}
            \item erlaubt Sicherheitseinstellungen beim Erstellen des REST-API-Servers
            \item API Key
            \item Authentication mit Passport
            \item explorer Test interface ?
            \item dynamic logging
            \item event publication over websockets
            \item TLS enable/\-disable
        \end{itemize}
    \end{itemize}
    
    \subsubsection{Konzept}
        \paragraph{\textrm{Model File}}\hspace{0cm}\smallskip\\
            Asset Types:
            \begin{itemize}[noitemsep]
                \item Ein Asset repräsentiert einen Schlüssel, um das Schloss zu öffnen (kann zeitlich beschränkt sein). 
                    Beim Öffnen des Schlosses wird das Asset mittels der Funktion ,,Schloss öffnen`` an Schloss gesendet.
                    Wird das Schloss wieder verriegelt, so sendet das Schloss den Token wieder an Nutzer zurück.
                    Somit kann sichergestellt werden, dass das Schloss geschlossen ist, wenn dieses 0 Token besitzt.
                \item Token, der die Rolle repräsentiert?
                \item Um \gls{dos} zu vermeiden, etwas ähnliches wie Mutex-Token? z.B. wenn Tür offen ist, dann hat sie genau diesen einen Token
                \item Assets können auch zu anderen Assets oder Teilnehmern eine Beziehung haben\cite{ComposerDocs}
            \end{itemize}
            
            Participant Types:
            \begin{itemize}[noitemsep]
                \item Manufacturer
                \item Owner
                \item Guest
            \end{itemize}
        
            Transaction Types (entpricht den Funktionen, die man je nach Rolle ausführen darf):
            \begin{itemize}[noitemsep]
                \item OpenDoor
                \item User (AddUser, DeleteUser, ChangeUserRole)
                \item TimeSlot (AddTimeSlot, DeleteTimeSlot, ChangeGuestTimeSlot)
            \end{itemize}
    
    \paragraph{Transaction Functions}
    
    \paragraph{Access Control Rules}
        \begin{itemize}
            \item Whitelisting
        \end{itemize}
    
    \paragraph{Query Definitions}
    
    Rollenbasiertes Zugriffskonzept
    \begin{itemize}[noitemsep]
        \item 
    \end{itemize}