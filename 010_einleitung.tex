\section{Einleitung}
\label{sec:intro}
    Mit stetig zunehmender Vernetzung des Lebens ist davon auch häufig der eigene Wohnraum betroffen. 
    Der Trend zu sogenannten Smart Homes ist klar erkennbar\cite{Paley2018}. 
    Von Küchengeräten über Beleuchtung, Sprinkleranlagen im Garten und Ga\-ra\-gen\-tü\-ren, immer mehr Geräte werden mit einem Netzwerk und gar mit dem Internet zu einem sogenannten \gls{iot} verbunden. 
    Gesteuert wird dies meist mit dem Smartphone entweder direkt oder über ein spezielles Hub, über das alle Informationen zentral fließen.
    So auch Smart Locks (wörtl. ,,intelligente Schlösser``).
    Diese werden häufig auch bei Buchung und Vermietung von privaten Unterkünften oder im eigenen Heim an der als Türschlosser oder auch in Form von Vorhängeschlössern eingesetzt.
    Sie sollen den Besitzern die Möglichkeit bieten das Schloss ohne physischen Schlüssel bequem mittels Smartphone zu öffnen und zu schließen\cite{Ho2016}.
    Häufig bieten Smart Locks auch Funktionen zur Administration von Berechtigungen, wie beispielsweise bestimmte Nutzer zeitweise dazu zu berechtigen das Türschloss zu öffnen und zu schließen\cite{Ye2017}.
    Oft wird zur Übertragung der Signale Bluetooth Low Energy verwendet\cite{Fuller2017}.
    \medskip\\
    Ebenfalls im Trend liegt die Technologie der Blockchain, welche mit dem Erfolg der Kryptowährung Bitcoin nun auch in anderen Gebieten wie im Internet of Things und im Smart Home Anwendung findet\cite{Conoscenti2016,Kshetri2017}.
    Da im Smart Home häufig auch kritische Daten, wie beispielsweise personenbezogene Daten ausgetauscht werden, ist deren Sicherheit zu garantieren wichtig.
    Ein zentrales Merkmal der Blockchain ist die Dezentralisierung der ,,Buchführung`` von Transaktionen.
    \medskip\\
    Aufgrund vermehrter Berichte über Sicherheitsvorfälle bei \gls{iot}-Geräten\cite{Goodin2018} ist es umso nötiger die Sicherheit der im \gls{iot} verarbeiteten Daten und die Funktion der vernetzten Geräte zu gewährleisten.
    Diese Berichte umfassen Schwachstellen wie statische Schlüssel, als Klartext gespeicherte Passwörter, die Möglichkeit für Replay-Angriffe, Device-Spoofing\cite{Rose2016} und ungesicherte APIs bei der Kommunikation mit der Cloud\cite{Stykas2018}. 
    Als eine der schwerwiegensten Schwachstellen wird außerdem die Zentralisierung von \gls{iot}-Geräten vor allem in der Cloud beschrieben\cite{Kshetri2017}.
    In Fachzeitschriften werden einige Herausforderungen, die durch eine Blockchain möglicherweise gelöst werden könnten, beschrieben und potentielle Lösungsansätze vorgestellt\cite{Kshetri2017}.
    \medskip\\
    Gerade bei Smart Locks ist es unbedingt nötig die vorhandenen Schwachstellen zu unterbinden, da diese oft die ,,Eingangstür`` zum Zuhause sind.
    Durch das dezentrale Konzept der Blockchain\cite{Nakamoto2008} lohnt es sich diese Technologie im Kontext des \gls{iot}, am Beispiel des Anwendungsfalls von Smart Locks zu untersuchen.
    
\newpage
\section{Problemstellung}
\label{sec:problem}
    Als Ziel der Arbeit soll die Frage erörtert werden, ob die Block\-chain\--Tech\-no\-lo\-gie aus dem Aspekt der Sicherheit dafür geeignet ist, im Bereich der Smart Locks eingesetzt zu werden.
    Dies soll mit Hilfe eines Prototypen eines Smart Locks untersucht und anhand von bereits aufgedeckten Schwachstellen verglichen werden.
    
    \subsection{Abgrenzung}
    \label{sec:problem_limit}
		Der Fokus des Prototypen liegt nicht auf der Umsetzung der Hardware, sondern auf der Nutzung eines aktuell vorhandenen Blockchain-Frameworks, also auf aktuell plausible Implementierungen.
		Somit wird im Ergebnis auch keine Aussage über die physische Sicherheit von Smart Locks gemacht.
		\medskip\\
		Es wird untersucht inwiefern und ob der Ersatz von herkömmlichen zentralen Komponenten durch eine Blockchain die Sicherheit erhöht. 
		Periphäre Komponenten wie Smartphone-Apps oder Webapplikationen zur Verwaltung von Smart Locks werden nicht betrachtet.
		\medskip\\
		Als Beispiel für herkömmliche Smart Lock Systeme dient, soweit nicht anders erwähnt, das August Smart Lock.

    \subsection{Methodik}
    \label{sec:problem_methods}
        Zunächst werden bekannte Schwachstellen aktueller Produkte analysiert.
        Als Leitfaden für die Analyse werden die \gls{owasp}-Top 10 für  \gls{iot}-Geräte\cite{Miessler2015a} verwendet.
        Im Anschluss wird als erstes ein passendes Framework nach bestimmten Aspekten ausgewählt.
        Auf Basis dieses Frameworks wird ein Prototyp mit minimalen Funktionsumfang entworfen und umgesetzt.
        Der Prototyp wird ebenfalls anhand der für die Analyse verwendeten \gls{owasp}-Top 10 für \gls{iot}-Geräte evaluiert.
        Danach wird zwsichen den beim Prototyp gefundenen und den zuvor bei aktuellen Produkten analysierten Schwachstellen verglichen.
        Der Vergleich wird durch ein standardisiertes Bewertungsschema, dem \gls{cvss} gestützt, welches eine Vergleichbarkeit zwischen den gefundenen Schwachstellen schafft.
        Abschließend wird die Problemstellung mittels des Vergleichs erörtert.
  
    \newpage
    \subsection{Vorhandene Arbeiten}
    \label{sec:problem_relatedWork}
        In \cite{Han2017} wird das Konzept eines Smart Locks von \cite{Park2009} mittels Blockchain-Technologie erweitert.
        Das Konzept basiert auf einer öffentlichen Blockchain und setzt den Fokus auf eine physische Umsetzung mittels zusätzlichen Sensoren, die potentielle Replay-Angriffe verhindern sollen.
        \medskip\\
        Es existieren zudem vorangegangene Sicherheitsuntersuchungen von Smart Locks \cite{Ho2016,Ye2017} und Smart Homes\cite{Fernandes2016}, die teilweise mit Fokus auf verschiedene Komponenten oder Kommunikationswege wie \gls{ble}\cite{Rose2016} oder einzelne Produkte (August Smart Lock\cite{Fuller2017}) durchgeführt wurden.
        \medskip\\
        Eine Fallstudie eines Smart Homes mit Blockchain-Technologie\cite{Dorri2017} macht Aussagen über potentielle Performance und mögliche Schwachstellen.
