\subsection{Internet of Things}
\label{sec:sota_iot}
    Am Ende mal schauen, was davon noch sinnvoll ist mit reinzunehmen... \todo[color=cyan]{Am Ende machen}
    \begin{itemize}
        \item Erklären, was das IoT ist
        \item Iot-Security?
        \item Smart Home?
        \item Protolle: (BLE, Z-WAVE, Zigbee)?
    \end{itemize}

    Security:
    Attack Surface Areas nach OWASP, nach die vorgestellten Schwachstellen eingeteilt werden:
    \begin{enumerate}[noitemsep]
        \item Insecure Web Interface
        \item Insufficient Authentication/Authorization
        \item Insecure Network Services
        \item Lack of Transport Encryption/Integrity Verification
        \item Privacy Concerns
        \item Insecure Cloud Interface
        \item Insecure Mobile Interface
        \item Insufficient Security Configurability
        \item Insecure Software/Firmware
        \item Poor Physical Security
    \end{enumerate}

\subsubsection{Kommunikationsprotokolle}
\label{sec:sota_iot_protocols}
    \cite{Gomez2012}
    In \cite{Rose2016}
    \begin{itemize}
        \item nur kleine Datenmengen können ausgetauscht werden (20Bytes)\cite{Rose2016}
        \item geriner Energieverbrauch
        \item funktioniert wie Bluetooth Classic auf 2,4Ghz
        \item kurze Reichweite (<100m)
        \item verschiedene Hosts mit jeweils verschiedenen Profilen, geht vom Controller aus
    \end{itemize}
    
    \paragraph{Profile}\cite{Rose2016}
        \begin{itemize}
            \item General Attribute Profile (Client schickt Anfrage an GATT Server, Server speichert Attribute)
        \end{itemize}
