\subsection{Double-Spending Problem}
\label{sec:sota_doublespend}
	Das Double Spending Problem beschreibt die Möglichkeit den selben digitalen Token, welcher als Gegenwert einem Käufer dem Kauf eines (digitalen) Guts dient, mehrmals auszugeben\cite{Chohan2017}.
	Die Gefahr kommt daher, da ein Token beispielsweise als eine Datei repräsentiert wird, welche dupliziert und gefälscht werden könnte\cite{Chohan2017}.
	Der Empfänger einer Transaktion kann die Echtheit und eventuell mit dem Token bereits getätigte Transaktionen im Normalfall nicht verifizieren\cite{Nakamoto2008}.
	Daher wird häufig eine \gls{ttp}, zum Beispiel von einer sogenannten Mint (primärer Produzent der Währung) übernommen zur Prüfung der Transaktion herangezogen\cite{Nakamoto2008}.
	Somit muss der Token zur Überprüfung an die Mint zurückgegeben werden. 
	Diese erzeugt einen neuen Token und gibt diesen an den Verkäufer weiter.
	Einzig diesem neuen Token kann vertraut werden, dass er nicht mehrfach genutzt wurde\cite{Nakamoto2008}.