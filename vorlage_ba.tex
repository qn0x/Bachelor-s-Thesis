\documentclass[toc=sectionentrywithdots,a4paper,11pt,oneside, openright]{scrartcl}

% Stilistische Vorgaben nach Standard der HS
%--------------------------------------------------------------------------
\usepackage{geometry}
\geometry{top=2.0cm,bottom=3cm,left=3.5cm,right=2cm}
%----------------Schriftart/-form------------------
%Serifen######
%-------------
%\usepackage{mathptmx} % setzt auf Times New Roman
%\renewcommand{\familydefault}{\rmdefault}
\setkomafont{section}{\fontfamily{\rmdefault}\Large}
\setkomafont{subsection}{\fontfamily{\rmdefault}\large}
\setkomafont{subsubsection}{\fontfamily{\rmdefault}\normalsize}
\setkomafont{sectionentry}{\fontfamily{\rmdefault}}

%Seriflos#####
%-------------
\renewcommand{\familydefault}{\sfdefault}
%\setkomafont{section}{\fontfamily{\sfdefault}\Large}
%\setkomafont{subsection}{\fontfamily{\sfdefault}\large}
%\setkomafont{subsubsection}{\fontfamily{\sfdefault}\normalsize}
%\setkomafont{sectionentry}{\fontfamily{\sfdefault}}
%---------------------------------------------------
\usepackage[T1]{fontenc} %
\usepackage{setspace}
\setcounter{tocdepth}{4}
%--------------------------------------------------------------------------
% Pakete für die Bearbeitung (Sprache, Tabellen, Grafiken, Mathe,...)
\usepackage[ngerman]{babel} % Sprachpaket
\usepackage[utf8]{inputenc} % Zeichenkodierung inkl. Umlaute
\usepackage{graphicx} % Einbinden von Bildern
\usepackage{longtable} % Tabellen die über eine Seite gehen
\usepackage{tabularx} % Standardtabellen
\usepackage{amsmath} % Für mathematische Zeichen, Formeln, etc...
\usepackage{tabto}
\usepackage{blindtext}
%Abkürzungsverzeichnis; Option: acronym trägt nur die Abkürzungen in das Verzeichnis ein. Kompilieren im TeXStudio über Tools-->Glossary-->dann gesamtest Dokument kompilieren
\usepackage[nonumberlist,nomain,acronym,xindy,automake,toc]{glossaries}
\makeglossaries
\loadglsentries{_glossary.tex}

%Kopf-/Fußzeile
%------------
\usepackage{fancyhdr}
\pagestyle{fancy}
\fancypagestyle{plain}{\fancyhf{}}
\fancyhf{}
\rfoot{\thepage}
\renewcommand{\headrulewidth}{0pt}
\renewcommand{\footrulewidth}{0pt}
%--------------------------------------------------------------------------
\usepackage[backend=biber,sorting=none]{biblatex}
\addbibresource{_bibliography.bib}

%-------------------------------------
% eigene pakete
\usepackage{textcomp}
\usepackage{hyperref}

%---------------------------------------------------
% PLATZHALTER FÜLLEN MIT NAMEN, THEMA, ETC.
\newcommand{\grad}{Bachelor of Science}
\newcommand{\matrinr}{s158109}
\newcommand{\thema}{Thema, wie beantragt.}
\newcommand{\name}{Janine Kostka}
\newcommand{\geb}{08.01.1994}
\newcommand{\ort}{Kirchheim unter Teck}
\newcommand{\erstp}{Prof. Benjamin Fabian}
\newcommand{\instituterst}{Hochschule für Telekommunikation Leipzig}
\newcommand{\zweitp}{M.Sc. Mario Hoffmann}
\newcommand{\institutzweit}{Hochschule für Telekommunikation Leipzig}
\newcommand{\abgabe}{31.10.2018}
%---------------------------------------------------


\begin{document}
	%---------------------------------------------------
	% TITELSEITE/DECKBLATT
	\begin{titlepage}
		\begin{figure}[h]
			\includegraphics[scale=0.2]{hftl_logo.png}
		\end{figure}
		\vspace*{20pt}
		\centering
		Hochschule für Telekommunikation Leipzig\\
		\vspace*{40pt}
		\large \textbf{Abschlussarbeit zur Erlangung des akademischen Grades}\\
		\doublespacing
		\textbf{\grad} % ggf. Grad anpassen
		\vspace*{100pt}
		\begin{table}[h!]
			\begin{tabular}{p{0.2\linewidth}p{0.7\linewidth}}
				Thema: & \large \thema \\
				\\[3em]
				& \large Serifenlose Schrift 
				\\[5em]
				Vorgelegt von: & \large \name  \\
				\\[2em]
				geboren am: & \geb \\
				in: & \ort \\
				Matrikelnummer: & \matrinr \\
			 	\\[2em]
			 	eingereicht am: & \abgabe \\
			 	\\[2em]
			 	Erstprüfer: & \erstp, \instituterst \\
			 	Zweitprüfer: & \zweitp, \institutzweit \\
			\end{tabular}
		\end{table}
	\end{titlepage}
	\newpage
	%---------------------------------------------------
	%---------------------------------------------------
	% SELBSTSTÄNDIGKEITSERKLÄRUNG
	\thispagestyle{empty}
	\vspace*{3em}
	\begin{center}
		\LARGE \textbf{Selbstständigkeitserklärung}
	\end{center}
	\normalsize
	\vspace*{3em}
	Hiermit erkläre ich, dass die von mir an der Hochschule für Telekommunikation Leipzig (FH)
	eingereichte Abschlussarbeit zum Thema
	\vspace*{1em}
	\begin{center}
		\thema
	\end{center}
	\vspace*{1em}
	vollkommen selbständig verfasst und keine anderen als die angegebenen Quellen und
	Hilfsmittel benutzt habe.
	\\[2em]
	Stellen, die wörtlich oder sinngemäß aus veröffentlichten oder noch nicht veröffentlichten
	Quellen entnommen sind, sind als solche kenntlich gemacht.
	\\[2em]
	Die Abbildungen in dieser Arbeit sind von mir selbst erstellt oder mit einem entsprechenden
	Quellennachweis versehen.
	\\[2em]
	Diese Arbeit ist in gleicher oder ähnlicher Form noch bei keiner anderen Hochschule/
	Universität eingereicht worden.
	\\[6em]
	
	Leipzig, den \abgabe \tab \rule{6cm}{0.5pt}\\
	\hspace*{22em}\name
	\newpage
	%---------------------------------------------------
	%---------------------------------------------------
	% VERZEICHNISSE -- AUTOMATISCHE EINTRAGUNG
	\normalsize
	\tableofcontents
	\newpage
	\listoffigures
	\newpage
	\listoftables
	\newpage
	\printglossary[type=\acronymtype]
	\thispagestyle{empty}
	\newpage
	%---------------------------------------------------
	%---------------------------------------------------
	%########################################################################################################
	%!!!!!!!!!!!!!!!!!!!!!!!!AB HIER ARBEITEN!!!!!!!!!!!!!!!!!!!!!!!!

    
\section*{Einführung/Motivation}
    % Was wollen Sie mit Ihrer Bachelorarbeit herausfinden? Was ist das Forschungsziel? Was wollen Sie erreichen? In diesem Kontext gehen Sie auch kurz auf die Ausgangslage ein: Wie ist der Forschungsstand zum Thema, welches Problem wird in Ihrer Bachelorarbeit behandelt werden und wo starten Sie mit der Untersuchung?
    
    Mit stetig zunehmender Vernetzung des Lebens ist davon auch häufig der eigene Wohnraum betroffen. 
    Der Trend zu sogenannten Smart Homes ist klar erkennbar. 
    Von Küchengeräten über Beleuchtung, Sprinkleranlagen im Garten und Ga\-ra\-gen\-tü\-ren - immer mehr Geräte werden mit einem Netzwerk und gar mit dem Internet verbunden. 
    Gesteurert wird dies meist mit dem Smartphone entweder direkt oder über ein spezielles Hub, über das alle Informationen zentral fließen.
    So auch Smart Locks (wörtl. ,,intelligente Schlösser``).
    Diese werden häufig auch bei Buchung und Vermietung von privaten Unterkünften oder auch im eigenen Heim eingesetzt und sollen den Besitzern die Möglichkeit bieten schlüssellos und bequem mittels Smartphone das Schloss zu öffnen und zu schließen.
    Häufig bieten Smart Locks auch Funktionen zur Administration von Berechtigungen, wie beispielsweise bestimmte Nutzer zeitweise dazu zu berechtigen das Türschloss zu öffnen und zu schließen. 
    
    Ebenfalls im Trend liegt die Technologie der Blockchain, welche mit dem Erfolg der Kryptowährung Bitcoin nun auch in anderen Gebieten wie im Internet of Things und im Smart Home Anwendung findet.
    Da im Smart Home häufig auch kritische Daten, wie beispielsweise personenbezogene Daten ausgetauscht werden, ist deren Sicherheit zu garantieren wichtig.
    Ein zentrales Merkmal der Blockchain ist die Dezentralisierung der ,,Buchführung`` von Transaktionen.

\section*{Problemstellung}
    % Anhand welcher Theorien, Methoden und Vorgehensweisen erledigt die Bachelorarbeit diese Aufgabe?
    Das Ziel der Arbeit soll die Frage erörtern ob die Block\-chain\--Tech\-no\-lo\-gie aus dem Aspekt der Sicherheit dafür geeignet ist, im Bereich der Smart Locks (und erweitert im Bereich Smart Home) eingesetzt zu werden.
    Dies soll mit Hilfe eines Prototypen eines Smart Locks untersucht werden.
    
    Der Fokus des Prototypen soll dabei aber nicht auf der Umsetzung der Hardware liegen, sondern auf der Nutzung eines aktuell vorhandenen Frameworks, also auf aktuell plausiblen Implementierungen. 
    Primär sollen bereits publizierte Schwachstellen bei Smart Locks analysiert werden und bei der Umsetzung des Prototypen vermieden werden. 
    Dies wird nach fertigstellung des Prototypen untersucht.
    Je nach Ergebnis lässt sich dann auf die Kernfrage schließen.

\section*{Stand der Wissenschaft}
    In dem Stand der Wissenschaft werden folgende Themen verschieden ausführlich abgedeckt:
    \begin{itemize}
        \item Blockchain (ausführlich): (eventuell kurze Historie, )Grundkonzepte, Anwendungsgebiete, Bezug auf Sicherheit
        \item \gls{iot} (kurz)
        \item Smart Home (kurz)
        \item Smart Locks (ausführlich): u.a. Architekturen, Sicherheitsmechanismen, ..
        \item Sicherheitsanalysen (von Smart Locks bzw. Geräten im Interet of Things) (ausführlich)
    \end{itemize}

\section*{Vorgehen}
    \subsection*{Auswahl des Frameworks}
        Ein (kurzer) Abschnitt der Arbeit soll die Auswahl des genutzen Frameworks für den Prototypen darstellen, da dies eine zentrale Entscheidung für die Arbeit und somit auch essentiell für die Bearbeitung der Problemstellung ist.
        Es existieren mittlerweile diverse Frameworks mit verschiedenen Anwendungsgebieten, daher sollte das letztendlich verwendete Framework folgenden Kriterien entsprechen:
        \begin{itemize}
            \item Open Source \textrightarrow\ eigene Einflussmöglichkeit
            \item aktive Community \textrightarrow\ Unterstützung bei Schwierigkeiten
            \item aktive Weiterentwicklung \textrightarrow\ zumindest in naher Zukunft besteht die Möglichkeit, dass das Framework auch in Produkten auf dem Markt genutzt werden könnte
            \item Auswahl an Sicherheitsmechanismen \textrightarrow\ Evaluierung von verschiedenen Konfigurationen möglich
        \end{itemize}
        \medskip
        \textbf{Auswahl verfügbarer Frameworks nach erster Recherche}
        \begin{itemize}
            \item Hyperledger (Auswahl an verschiedenen Projekten mit unteschiedlichen Einsatzszenarien):\\
                Favorit: Iroha - mit rollenbasierter Rechtevergabe und Unterstützung für mobile Plattformen, sowie Bibliotheken für mehrere Sprachen und guter Dokumentation
            \item Exonum
            \item Ethereum
            \item Multichain
        \end{itemize}

    \subsection*{Analyse von aktuellen Produkten}
        Die Analyse von aktuellen Produkten soll Teil der Literaturrecherche sein. 
        Dabei werden beispielsweise die Case Study eines Smart Locks von der Firma August\cite{Ye2017}, sowie Nachrichtenartikel\cite{Tsing2017} und Präsentationen von Konferenzen\cite{Rose2014} herangezogen.
        Aus die in dieser Analyse identifizierten spezifischen Schwachstellen bei Smart Locks sollen die Anforderungen an den Prototypen abgeleitet werden.
    \subsection*{Prototypische Umsetzung}
        Je nach Aufwand und Zeigbudget kann der Prototyp in seiner Funktion variieren. 
        Folgend die Vorstellungen der Mindestfunktionen des Prototyps.
        \begin{itemize}
            \item Hinzufügen von neuen Nodes/Peers
            \item Funktion ,,Schloss öffnen``
            \item Vergeben/Entziehen von Rechten die Funktion ,,Schloss öffnen`` zu nutzen
            \item Übertragung der Daten zwischen Nodes via Bluetooth oder anderem Protokoll
        \end{itemize}
    
    \subsection*{Evaluation des Prototypen}
        % Was könnte bei Ihrer Bachelorarbeit als Ergebnis stehen? Hier geht es darum, zu reflektieren, was eine realistische Ziel- und Ergebniskombination ist, die innerhalb des vorgegebenen Bearbeitungszeitraumes erreicht werden kann.
        \subsubsection*{Vorgehen bei der Identifizierung von Schwachstellen}
            Der Prototyp soll auf die im voraus identifizierten Schwachstellen überprüft werden.
            Jedoch existieren generell eher wenige Sammlungen an Informationen über das Testen von Geräten im \gls{iot}.
            Das \gls{owasp} hat jedoch bereits eine Auflistung möglicher Angriffsflächen, sowie einen Testing Guide veröffentlicht, welche als Grundlagen für die Sicherheitsuntersuchung genutzt werden sollen.
            Damit soll eine breitgefächerte und einheitliche Teststruktur garantiert werden.


        \subsubsection*{Bewertung der Testergebnisse}
            Die Testergebnisse sollen nach dem \gls{cvss} V3.0 bewertet werden.


        \subsubsection*{Beantworten der Fragestellung}
            Es soll schließlich im Bezug auf die Problemstellung mit den Ergebnissen argumentiert werden.
            Mögliche Ergebnisse der Frage, ob die Blockchain-Technologie geeignet wäre in Smart Locks eingesetzt zu werden:
            \begin{enumerate}
                \item Ja, keine Schwachstellen gefunden oder die gefundenen Schwachstellen haben (vermutlich) keinen Einfluss auf die Sicherheit des Smart Locks.
                \item Ja, mit Nachbesserungen zu einem eventuell späteren Zeitpunkt oder einem anderen Framework.
                \item Nein, das Konzept der Blockchain lässt sich nicht mit dem Anwendungsfall eines Smart Locks vereinbaren.
                \item Nein, die gefunden Schwachstellen sind so elementar, dass es unzumutbar ist zumindest das ausgewählte Framework zu nutzen.
            \end{enumerate}


\section*{Erwarteter Nutzen der Ergebnisse}
    Im Großen und Ganzen existieren bisher eher wenige Veröffentlichungen über Smart Locks.
    Im Paper von \cite{Han2017} wurde ein mögliches Konzept eines Blockchain-basierten Smart Locks vorgestellt, jedoch ist dies jeglich eine theoretische Abhandlung ohne Überprüfung des Ergebnisses.
    
    Dadurch, dass es kritisch ist seine Eingangstür zu Verschließen, ist ein Schloss ein guter Repräsentant für die Sicherheit im Smart Home bzw. im Internet of Things. 
    Das erarbeitete Konzept lässt sich auf weitere Anwendungen mit gleichen oder ähnlichen Sicherheitsanforderungen (Identität, Authorisierung, Authentifizierung, Vertraulichkeit, ...)in den o.g. Gebieten übertragen.

\section*{Planung}
    \subsection*{Termine}
        \begin{itemize}
        %% Bearbeitungszeitraum (regulär) 12 Wochen
            \item Beginn: KW35/36
            \item Abgabe: 2.11.18 (Ende KW44)
            \item spätestes Enddatum: 30.11.18 (KW48)
            \item Kolloquium: am liebsten zwischen Oktober/November 
            Da ich nach Ende des Studiums eine Weiterbildung zum Cyber Secrutiy Professional machen werde, werde ich für ein Modul am 8.10, sowie am 5.11. wohl ganztägig an der HfTL sein. Für das Kolloquium werden wir freigestellt, was bedeutet, dass sich die darauffolgenden Tage 9.10. und 6.11. für mich als Zeitpunkt anbieten würden. Andernfalls bin ich terminlich nicht eingeschränkt.
        \end{itemize}

    \subsection*{Zeitplanung}
        \begin{itemize}
            \item evtl. 1 Woche Vorbereitung (Literaturrecherche, Einarbeitung ins Framework)
            \item 2 Wochen Prototyp entwickeln und testen
            \item 4 Wochen Inhalt der Arbeit schreiben
            \item 1-2 Wochen Puffer für Korrekturen und Verbesserungen
        \end{itemize}

    \subsection*{Vorläufige Gliederung}
        \begin{enumerate}
            \item Einführung
            \item Stand der Wissenschaft (mit den o.g. Unterpunkten)
            \item Analyse vorhandener Produkte
            \item Konzipierung und Implementierung des Prototypen
            \item Evaluation des Prototypen
            \item Fazit
            \item Schluss
            \item Anhang: Auszüge aus der Implementierung des Prototypen
        \end{enumerate}

%%   Fragen:
%% - Ist die Anzahl der Themen im Stand der Wissenschaft zu viel oder noch zu wenig?
%% - Welche Codeschnipsel aus dem Prototypen sind relevant und sollten als Anhang in der Arbeit sein?
%% - Titel so okay oder muss noch angepasst werden?
%% - Weitere mögliche Ergebnisse als die bereits aufgelisteten?

\nocite{*}
    \section{Einleitung}\todo[color=cyan]{Vorwort und Abstract schreiben}
    Mit stetig zunehmender Vernetzung des Lebens ist davon auch häufig der eigene Wohnraum betroffen. 
    Der Trend zu sogenannten Smart Homes ist klar erkennbar. 
    Von Küchengeräten über Beleuchtung, Sprinkleranlagen im Garten und Ga\-ra\-gen\-tü\-ren - immer mehr Geräte werden mit einem Netzwerk und gar mit dem Internet zu einem sogenannten \gls{iot} verbunden. 
    Gesteurert wird dies meist mit dem Smartphone entweder direkt oder über ein spezielles Hub, über das alle Informationen zentral fließen.
    So auch Smart Locks (wörtl. ,,intelligente Schlösser``).

    Diese werden häufig auch bei Buchung und Vermietung von privaten Unterkünften oder im eigenen Heim an der als Türschlosser oder auch in Form von Vorhängeschlössern eingesetzt und sollen den Besitzern die Möglichkeit bieten schlüssellos und bequem mittels Smartphone das Schloss zu öffnen und zu schließen.
    Häufig bieten Smart Locks auch Funktionen zur Administration von Berechtigungen, wie beispielsweise bestimmte Nutzer zeitweise dazu zu berechtigen das Türschloss zu öffnen und zu schließen.
    Oft wird zur Übertragung der Signale Bluetooth Low Energy verwendet.
    
    Ebenfalls im Trend liegt die Technologie der Blockchain, welche mit dem Erfolg der Kryptowährung Bitcoin nun auch in anderen Gebieten wie im Internet of Things und im Smart Home Anwendung findet.
    Da im Smart Home häufig auch kritische Daten, wie beispielsweise personenbezogene Daten ausgetauscht werden, ist deren Sicherheit zu garantieren wichtig.
    Ein zentrales Merkmal der Blockchain ist die Dezentralisierung der ,,Buchführung`` von Transaktionen.
    \newline
    
    \noindent Aufgrund vermehrter Berichte über Sicherheitsvorfälle bei \gls{iot}-Geräten ist es umso nötiger die Sicherheit der im \gls{iot} verarbeiteten Daten und die Funktion der vernetzten Geräte zu gewährleisten.
    Diese Berichte umfassen Schwachstellen wie hardcoded Schlüssel, im Klartext gespeicherte Passwörter, Möglichkeit für Replay-Angriffe, Device-Spoofing\cite{Rose2016} und ungesicherte APIs beim Kommunikation mit der Cloud\cite{Stykas2018}. 
    Als eine der schwerwiegensten Schwachstellen wird außerdem die Zentralisierung von \gls{iot}-Geräten vor allem in der Cloud beschrieben\cite{Kshetri2017}.
    \newline\smallskip
    
    \noindent\textbf{Motivation}\todo[color=cyan]{motivation schreiben}\newline
    The key thing to keep in mind is this: if you have a set of users (a) who want to trade digital tokens, and (b) have agreed on how these tokens are generated, then a blockchain network is an ideal tool to use both for exchanging these tokens, and tracking who has what. No middleman is needed to facilitate the exchanges cause every node on the network runs the the necessary checks and reaches consensus on the accepted result. Asset tracking comes out-of-the-box since every node has access to the agreed set of cryptographically verifiable transactions on the blockchain.\cite{Christidis2016}
    \newline\smallskip
    
\section{Problemstellung}
    %Anhand welcher Theorien, Methoden und Vorgehensweisen erledigt die Bachelorarbeit diese Aufgabe?
    Gerade bei Smart Locks ist es unbedingt nötig diese Schwachstellen zu unterbinden.
    Durch das oben erwähnte dezentrale Konzept der Blockchain\cite{Nakamoto2008} lohnt es sich diese Technologie im Kontext des \gls{iot}, am Beispiel von Smart Locks zu untersuchen.
    Zudem verspricht man sich auch im Bereich des Identitäts- und Zugriffsmanagements von dem Konzept Blockchain Angriffe wie Device-Spoofing durch das Speichern von Gerätesignaturen zu unterbinden\cite{Kshetri2017}.
    \newline
    \noindent Als Ziel der Arbeit soll die Frage erörtern ob die Block\-chain\--Tech\-no\-lo\-gie aus dem Aspekt der Sicherheit dafür geeignet ist, im Bereich der Smart Locks (und erweitert im Bereich Smart Home) eingesetzt zu werden.
    Dies soll mit Hilfe eines Prototypen eines Smart Locks untersucht werden.
    
    Der Fokus des Prototypen soll dabei aber nicht auf der Umsetzung der Hardware liegen, sondern auf der Nutzung eines aktuell vorhandenen Frameworks, also auf aktuell plausiblen Implementierungen. 
    Primär sollen bereits publizierte Schwachstellen bei Smart Locks analysiert werden und bei der Umsetzung des Prototypen vermieden werden. 
    Dies wird nach fertigstellung des Prototypen untersucht.
    Je nach Ergebnis lässt sich dann auf die Kernfrage schließen.
    
    \subsection{Abgrenzung}

    \subsection{Methodik}
        Zunächst sollen bekannte Schwachstellen aktueller Produkte analysiert werden.
        Als roter Faden der Analyse werden die \gls{owasp}-Top10 für das \gls{iot}\cite{Miessler2015a} verwendet.
        Im Anschluss wird als erstes ein passendes Framework ausgewählt, welches theoretisch die in der Analyse gefundenen Lücken schließen könnte.
        Auf Basis dieses Frameworks wird dann der Prototyp entworfen und umgesetzt.
        Der Prototyp soll ebenfalls anhand der \gls{owasp}-Top10 evaluiert werden.
        Danach wird zwsichen den beim Prototyp gefundenen und den zuvor bei aktuellen Produkten analysierten Schwachstellen verglichen.
        Dies geschieht mit Hilfe des \gls{cvss}-Bewertungsschemas, welches eine Vergleichbarkeit zwischen den gefundenen Lücken schaffen soll.
        Abschließend wird die Problemstellung mittels des Vergleichs erörtert.
    
    \subsection{Vorhandene Arbeiten}

	\vspace{3em}    
	\noindent In \fref{sec:sota} werden zunächst die Grundlagen für diese Arbeit vorgestellt, darunter das Konzept einer Blockchain in \fref{sec:blockchain_introduction}, das Double-Spending Problem in \fref{sec:blockchain_doublespend} und einige Sicherheitsaspekte in \fref{sec:blockchain_security}.
	Weiterhin werden \gls{iot}(\fref{sec:iot}), Smart Home(\fref{sec:smart_home}) und Smart Locks(\fref{sec:smart_locks}) erklärt\todo[color=orange]{wdh}.
	Auf Sichereitsanalysen im \gls{iot} wird ausführlicher eingegangen.
	\todo[color=yellow]{kurzen Überblick über die Arbeit schreiben}

	
    
    \section{State of the Art}
    \begin{itemize}
        \item Blockchain (ausführlich): (eventuell kurze Historie, )Grundkonzepte, Anwendungsgebiete, Bezug auf Sicherheit, Identity Management, Self-Sovereign Identitiy
        \item \gls{iot} (kurz)
        \item Smart Home (kurz)
        \item Smart Locks (ausführlich): u.a. Architekturen, Sicherheitsmechanismen, ..
        \item Sicherheitsanalysen (von Smart Locks bzw. Geräten im Interet of Things) (ausführlich)
    \end{itemize}
    
    
\subsection{Blockchain}
    2008 von Satoshi Nakamoto erfunden, ursprünglich als Peer-to-Peer Electronic Cash System ,,Bitcoin``.
    
    erste digitale Währung, die 
    ohne zentrale Authorität bzw. zentralen Server das double-spendung Problem lösen sollte\cite{Nakamoto2008}. 
    
    Double Spending Problem: beschreibt die Möglichkeit den gleichen Token nochmal auszugeben.
    
    Dies wird mittels kryptographischem Beweis anstatt Vertrauen gelöst.
    \section{Analyse}
\label{sec:analysis}
	
	Um die spezifischen Anforderungen für den Prototypen festlegen zu können, werden zunächst bekannte Sicherheitslücken von bestehenden Produkten analysiert. 
	Für die Vergleichbarkeit der Evaluation am Ende werden die Lücken kategorisiert und nach dem \gls{cvss} bewertet.
	Die Bewertung wird mittels des bereitgestellten Rechners durchgeführt \url{https://www.first.org/cvss/calculator/3.0}.
	\todo[color=cyan]{Kategorisierung erklären, nur Base Score verwendet}

\subsection{Bekannte Sicherheitslücken und Angriffe}
\label{sec:analysis_weaknesses}

	Schwachstellen und Sicherheitslücken in August Smart Lock nach \citeauthor{Ye2017}.
	Setzt u.U. voraus, dass der Angreifer Zugriff auf den mobilen Client des Benutzers hat und evtl. Jailbroken/rooted ist.
	In der Applikation werden nach den Untersuchungen von \citeauthor{Ye2017} ein Handshake-Key, Benutzeraccount und persönliche Informationen als Klartext innerhalb einer XML-Datei abgelegt.
	\begin{itemize}
		\item nutzt direktes Bluetooth-Pairing, welches nur auf dem konstanten/statischen Handshake-Key basiert
		\item keine Authentifizierung der Pairing-Partner zwischen Mobile Client und Lock
		\item Kopieren von der Einstellungen aus dem System von Alice.
		    Eve legt sich einen neuen Account an und kopiert die Einstellungen von Alice, sodass von ihrem Nutzerkonto auf diese zugegriffen wird. \textrightarrow Mandatory Access Control
		\item Personenbezogene Daten des Nutzers liegen in Klartext auf dem Dateisystem
		\item DOS: Wenn mehrere Nutzer gleichzeitig versuchen sich mit dem Smart Lock zu verbinden wird bei einem die App suspendiert, sodass niemand gleichzig Öffnen/Schließen-Befehle senden kann.
		    Somit kann ein Angreifer kontinuierlich wechseln Öffnen/Schließen-Befehle senden und der legitime Nutzer kann die Smart Lock nicht kontrollieren.
	   \item Angreifer repliziert die App und bringt den Nutzer dazu die bösartige App zu verwenden, da keine Authentifizierung zwischen App und Nutzer
	   \item Bluetooth interfering
	   \item OS Level flaws
	\end{itemize}

	Andere Locks
	\begin{itemize}
		\item Kwikset Kevo smart lock but the physical lock contains serious vulnerabilities making it easily compromised, which only takes 10 seconds\cite{Ye2017}
		\item QuickLock Smart Lock does not encrypt the passwords and sends the password to the user who forgets the password in plaintext\cite{Ye2017}
		\item iBluLock Smart Lock 6-character password, which is vulnerable against the brute forcing attack\cite{Ye2017}
	\end{itemize}
	
	Weitere im Paper von \citeauthor{Ye2017} aufgelistete Angriffe\todo[color=cyan]{sortieren und filtern}
	\begin{itemize}
	    \item August smart lock (earlier Version) has hard-coded secret key in the application source code \cite{Rose2016}
		\item August: does not perform the 2-factor authentication properly, and the hackers compromising the user email and text message could illegally control the lock
		\item does not perform the password reset process properly, and the	attackers can easily figure out the true verification code for resetting any passwords
		\item Physically-present Attack: Physically performing the attack for the user	who forgets to lock the smart lock
		\item Revoking Attack: Performing the attack from the user who had the legal accessing before, such as the Airbnb tenant, or the household worker
		\item Stealing Attack: Performing the attack as the thief, and stealing the user device to control the smart lock
		\item Relaying Attack: Performing the attack by two attackers to relay the data for interfering with the smart lock control
		\item Bluetooth Jammer \textrightarrow DOS
		\item Mobile Device: Offizielle App fälschen oder bösartige App verwenden \textrightarrow Informationen des Nutzers stehlen \textrightarrow 
		\item Handshake Key Leakage Attack: in which the attacker is able to steal the handshake key from the smart lock, and	illegally and covertly control the lock using a third-party	device\\
		    Kann einfach durch ein vom Hersteller öffentliches Open Source Programm ausgenutzt werden ohne für den Besitzer sichtbare Spuren zu hinterlassen (und ohne die offizielle App zu benutzen)
		\item Owner Account Leakage Attack: in which the attacker is able to disguise himself/herself to be the owner, by logging into the lock owner’s account in the third-party device to control the smart lock without being discovered
		\item Personal Information Leakage Attack: in which the attacker is able to obtain the lock user information, which seriously threatens the user privacy; and
		\item Denial-of-service (DoS) Attack: in which the attacker disrupts the regular usage of smart lock, which dramatically	brings down the user experience
	\end{itemize}

\subsection{Bewertung nach CVSS}
\label{sec:analysis_cvss}
	Tabelle mit den Angriffen, kategorisiert, ähnliche Angriffe zusammengefasst.
	
	
\subsection{Ableiten der Anforderungen an den Prototypen}
\label{sec:analysis_requirements}
    Aus den in \fref{sec:analysis_cvss} gefundenen Schwachstellen und Sicherheitslücken lassen sich direkt folgende Anforderungen ableiten, die diese vermeiden:
    \begin{itemize}
        \item Sicherung der (relevanten) Daten, bevor diese auf dem Mobilgerät gespeichert werden\cite{Ye2017}
        \item sicheres Kommunikationsprotokoll für die Dateiübertragung via Bluetooth\cite{Ye2017}
        \item Authentifizierung des Nutzers und des Smart Locks, um sicherzustellen, dass niemand anderes den zwischen beiden Parteien steht (Man-In-The-Middle)\cite{Ye2017}
        \item Mandatory Access Control für die Systeme, Daten auf den Systemen und Funktionen innerhalb eines Systems.\cite{Ye2017}
        \item Priorisierung der Befehle\cite{Ye2017} je nach Rolle.
        \item evtl: Authentifizierung des Clients gegeüber dem Nutzer und umgekehrt.
    \end{itemize}
	
	Weiterhin lassen sich mittels den Ressourcen des \gls{owasp}-Projekts\cite{Miessler2015}\cite{Miessler2015a} weitere Anforderungen aufstellen\todo[color=orange]{besser ausdrücken}:
	\begin{itemize}
	    \item 
	\end{itemize}
    \subsection{Prototyp}
\subsection{Auswahl des Frameworks}



\subsubsection{Architektur und Funktionsweise}
    Enstprechend dem Kapitel mit dem Stand der Wissenschaft wurden die Konzepte im Framework umgesetzt.
    
    \include{04_evaluation}
    \section{Schluss}
\subsection{Ausblick}
	Ideen zur Erweiterung des Prototypen
	\begin{itemize}
		\item Self-Sovereign Identitiy
		\item Decentralized Identifiers
	\end{itemize}

	%!!!!!!!!!!!!!!!!!!!!!!!!BIS HIER ARBEITEN!!!!!!!!!!!!!!!!!!!!!!!!
	%########################################################################################################
	
	
\newpage	
\printbibliography[title=Literaturverzeichnis]
\end{document}