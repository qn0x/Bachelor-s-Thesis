\section*{Fragen}
    Es gibt einen Unterschied zwischen Minern und Teilnehmern (und das auch noch je nach Blockchain-Framework..). Darüberhinaus: Die Sicherheit steht und fällt nicht ganz mit der Mehrheit von ehrlichen Nutzern. In der Basis-Blockchain geht darum, einen Nonce-Wert zu erraten um einen besonders gearteten Hashwert herauszubekommen. Es ist nur statistisch unwahrscheinlicher dass eine Minderheit von unehrlichen Minern dieses Wert zuerst herausbekommt als die ehrlichen. Ferner ist es statistisch noch unwahrscheinlicher das gleich zweimal am Stück zu schaffen und somit die „längste Kette“ zu formen (darum geht es ja: vergangene Transaktionen zu verändern à erfordert „aufholen“ zur aktuellen Länge à „selfish mining“). Die Statistik spricht für die Praxis und bestätigt diese, daher wird gesagt „es ist genau deshalb sicher“…allerdings ist es mit Vorsicht zu genießen.
    
    \begin{itemize}
        \item Vor Abgabe noch Sophie schicken und ,,wichtige`` Kapitel markieren
        \item Wichtige Kapitel: 1 (Einleitung), 2 (Problemstellung)
    \end{itemize}
    
\section*{ToDo}
    \begin{itemize}[noitemsep]
        \item Abstract schreiben
        \item Anhänge schreiben
        \item Source Code Listings stylen
    \end{itemize}
    
\begin{table}[H]
\centering
\begin{tabular}{l|c|c|c}
                               & \textbf{Seitenanzahl (geplant)} & \textbf{Seitenanzahl (ist)} & \textbf{Status}        \\ \hline
\textbf{Einleitung}            & 1 - 3                           & {\color{yellow} 1}          & {\color{green} Done}   \\ \hline
\textbf{Problemstellung}       & 1 - 2                           & {\color{green} 2}           & {\color{green} Done}   \\ \hline
\textbf{Grundlagen}            & 10 - 15                         & {\color{yellow} 18}         & {\color{green} Done}   \\ \hline
\textbf{Analyse}               & 5 - 10                          & {\color{green} 7}           & {\color{green} Done}   \\ \hline
\textbf{Prototyp}              & 15                              & {\color{red} 8}             & {\color{green} Done}   \\ \hline
\textbf{Evaluation}            & 5 - 10                          & {\color{red} 1}             & {\color{yellow} Doing} \\ \hline
\textbf{Vergleich}             & 3                               & {\color{red} 1}             & {\color{red} Todo}     \\ \hline
\textbf{Schluss}               & 2 - 3                           & {\color{red} 1}             & {\color{red} Todo}     \\ \hline
\textbf{Gesamt}                & 44 - 61                         & {\color{yellow} 42}         & {\color{red} Todo}     \\
\end{tabular}
\end{table}