\section*{Fragen}
    \begin{itemize}
        \item Soll ich alle Quellen mit abgeben, an die man nicht so einfach kommt(Paywalls)? Wenn ja, wie (als Archiv)?
        \item Soll auch in irgendeiner Form eine Anleitung zur Nutzung des Prototypen dabei sein, bspw. im Anhang?
        \item Soll die Beschreibung der Methodik des Vorgehens immer häppchenweise am Anfang eines Kapitels stehen und in der Einleitung ein kurzer Ablauf der Arbeit?
        \item Erklärung des verwendeten Bewertungsschemas für die gefundenen Schwachstellen: soll diese schon bei den Grundlagen stehen oder erst bei der Analyse selbst erläutert werden?
        \item Sollen die verwendeten Akronyme in Verzeichnis geordnet sein? Wenn ja, Alphabetisch, nach Auftreten im Text,...?
        \item Die im ILIAS vorgegebenen Layoutvorlagen stimmen nicht mit der dort abgelegten Latex-Vorlage überein. Welche sollen nun verwendet werden?
    \end{itemize}
    
\section*{How to todos:}
    \begin{itemize}
        \item \textcolor{cyan}{fehlender Inhalt}
        \item \textcolor{yellow}{Unsicher}
        \item \textcolor{orange}{besser ausdrücken}
        \item \textcolor{red}{falsch}
        \item \textcolor{green}{Korrekturlesen}
    \end{itemize}
    
\begin{table}[H]
\centering
\begin{tabular}{l|c|c|c}
                               & \textbf{Seitenanzahl (geplant)} & \textbf{Seitenanzahl (ist)} & \textbf{Status}        \\ \hline
\textbf{Einleitung}            & 1 - 3                           & {\color{yellow} 1}          & {\color{green} Done}   \\ \hline
\textbf{Problemstellung}       & 1 - 2                           & {\color{green} 2}           & {\color{green} Done}   \\ \hline
\textbf{Grundlagen}            & 10 - 15                         & {\color{green} 14}          & {\color{yellow} Doing} \\ \hline
\textbf{Analyse}               & 5 - 10                          & {\color{green} 7}           & {\color{green} Done}   \\ \hline
\textbf{Prototyp}              & 15                              & {\color{red} 8}             & {\color{yellow} Doing} \\ \hline
\textbf{Evaluation}            & 5 - 10                          & {\color{red} 1}             & {\color{red} Todo}     \\ \hline
\textbf{Vergleich}             & 3                               & {\color{red} 1}             & {\color{red} Todo}     \\ \hline
\textbf{Schluss}               & 2 - 3                           & {\color{red} 1}             & {\color{red} Todo}     \\
\end{tabular}
\end{table}